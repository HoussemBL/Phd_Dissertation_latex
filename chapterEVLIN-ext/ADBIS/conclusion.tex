%This paper presents an approach that leverages evolution provenance collected from previous users 'exploration sessions in order to generate globally interesting trends known as collaborative-filtering recommendations. We contributed initially several methods to merge appropriately individual evolution provenance graphs. Experiments show that our proposed merger method,  \emph{MLM-Best Distance} outperforms other methods and produces compact multi-users graphs.
%Then, we describe a baseline collaborative-filtering method and several optimizations aiming at computing top-k recommendations. Experiments show the effectiveness of our optimizations that speed up significantly the recommendation computation time.



This paper extends our visual interactive data exploration framework {\color{Fuchsia}\framework{}} by a novel, collaborative-filtering recommendation approach that leverages evolution provenance collected from many previous users 'exploration sessions. 
In particular, it discussed how to merge graphs representing individual exploration sessions into a multi-user graph. This graph is then used to compute and rank query recommendations, for which we proposed several alternative algorithms as well. Experiments validated that our optimized merge and recommendation algorithms are both efficient and effective for the considered task. 

While this paper presents a first practical solution that combines collaborative-filtering and content-based recommendations for visual data exploration, we intend to study in more depth further combination strategies of the individually obtained scores. Second, we plan to investigate further methods susceptible to improve the quality of recommendations such as matrix factorization techniques.
%will investigate other collaborative-filtering techniques such as matrix factorization in order 


%  initially several methods to merge appropriately individual evolution provenance graphs. Experiments show that our proposed merger method,  \emph{MLM-Best Distance} outperforms other methods and produces more compact multi-users graphs.
%Then, we described a baseline collaborative-filtering recommender method and several optimizations. Experiments show the effectiveness of our optimizations that speed up significantly the recommendation computation time.
%Finally, case studies show that our proposed recommender quantifier that blends collaborative-filtering and based-content recommendation contributes to a more efficient exploration experience by providing users with diverse recommendations with distinguishable interestingness scores.



%Investigating further combination functions is left for future work.

%Invetigation of other graphs matching methods a part stable matching In this paper, we opt for a stable marriage approach, implementation and evaluation using further interpretations is left as future work.