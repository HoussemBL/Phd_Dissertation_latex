 \section{Conclusion}
 
In this chapter, we presented our collaborative-filtering recommendation approach proposed to improve the user experiment when exploring visually data warehouses.
%In particular, we discussed our provenance-based recommendation techniques including the content-based query recommendation, the collaborative-filtering query recommendation and the visualization recommendation.
{\color{Fuchsia}
In particular, we discussed how previous exploration sessions made by several users are merged into one multi-user graph that reflects global trends.
Afterthat, we present our approach that leverages the multi-user graph to compute collaborative-filtering recommendations.}
Furthermore, we discussed how our collaborative-filtering approach is leveraged to adjust the interestingness scores of generated recommendations.

 
Finally, we perform a thorough experimental evaluation to assess our proposed techniques as well as to better understand the overall system behavior of {\color{Fuchsia}our  framework \framework{} that gathers} all discussed scientific contributions discussed in this chapter.
Overall, our contributions were evaluated  both quantitatively using performance measurements and qualitatively with a user study on both synthetic and real data. 
Quantitative experiments show the feasibility and the efficiency of our proposed solutions for visual and interactive data exploration while qualitative evaluations show a general satisfaction among users when visually exploring data using the extended version of our framework {\color{Fuchsia}\framework{}}. 
  
  
  
  The evaluations showed promising results.  It opens also the path to a variety of research questions, which we intend to tackle in the future. For instance, our evolution provenance merge was promising 
 and paves the way to the support of the collaborative-filtering query recommendation.
  Accordingly, we consider that provenance aggregation may serve other applications apart from collaborative-filtering query recommendation.
   To this end, we worked also on the aggregation of several types of provenance. This results in a new provenance summary approach thoroughly discussed in the next chapter.




