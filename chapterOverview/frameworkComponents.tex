This section describes the modules and components present in the architecture of our visual data exploration framework as depicted in Figure~\ref{fig:archi-FW}.

~~\\
\noindent \textbf{Data.}
The \emph{Data} presents the input of our framework \framework{}.
It is structured data that may reside in a relational database, flat files, etc$\ldots$. This input requires proper access methods to retrieve subsets of the data.
 %Several examples can be used as a data input to our framework. This includes for instance, relational databases, data cubes, graphs, $\ldots$

~~\\
\noindent \textbf{Visual user interface.}
The \emph{Visual user interface component} allows users to interact with data of interest (e.g.,  through queries, interactions, faceted navigation, $\ldots$).
It consists of two main interfaces where the first one renders data regions under investigation. The second interface encompasses recommendations suggested to the user to explore next.

To provide these two interfaces, the \emph{Visual user interface module} consists of two sub-components. The first one is the \emph{Data display} sub-component. 
It takes as input a query $Q$ specified by the user. In this case, the \emph{Data display} sends the query
to the \emph{Data retrieval module} in order to process it and to get results.
Later, the \emph{Data display} sub-component receives the users' query result (a dataset $D_{explo}$ and a visualization specification $V_{explo}$) from the \emph{Data retrieval module}.
Ultimately, it renders $D_{explo}$ according to the visual encodings properties specified in the visualization $V_{explo}$.
The second sub-component is the \emph{Recommendation display} sub-component. It takes as input a set of triples (a recommended query $Q_{rec}$, a recommended visualization $V_{rec}$ and a score of interestingness $sc$).
Ultimately, it renders an interactive visualization $VIS$ that shows the set of recommendations that the user can study next.
Note that the \emph{Visual user interface} component is interactive. Hence, the user can issue queries whose results will be rendered by the \emph{Data display} sub-component. Furthermore, a user can inspect interactively results rendered by the \emph{Recommendation display} sub-component to gain more knowledge (e.g., via hovering or clicking on sub-results of interest). Overall, possible user interactions and navigations are described in the following Section~\ref{sec:frame-navi}.




~~\\
\noindent \textbf{Data retrieval.}
%The \emph{Data retrieval module} is responsible for processing users' requests to fetch a specific region of data. 
%It takes as input a dataset $D$ and a query $Q$. 
%Then, it returns $Q(D)$, the subset of data $D$ that meets the user's request specified in $Q$.
{\color{Fuchsia}
The \emph{Data retrieval module} is responsible for processing users' requests to fetch a specific region of data. 
It takes as input a dataset $D$ and a query $Q$ (specified via the \emph{Data display} sub-component
).
Then, it returns a pair of information: (i) $D_{explo}$, a region of data $ \subset D$ that meets the user's request $Q$, and (ii)  $V_{explo}$ (sent from the \emph{Recommendation display} sub-component) that contains the visual encodings properties required to render $D_{explo}$.
}

~~\\
\noindent \textbf{Provenance engine module.}
The \emph{Provenance engine module} is responsible for collecting/processing provenance information that tracks an actual run of a visual data exploration process.
It takes as input information about explored data region $D_{explo}$ from the \emph{Data retrieval module}. 
%Collected provenance encompasses meta data about the visual data exploration process such as user's interactions, user's exploration queries, and inspected visualizations. 
{\color{Fuchsia}The \emph{Provenance engine module} takes also as input visual properties information $V_{explo}$ used to render data region $D_{explo}$.
Pairs of explored data region $D_{explo}$ and visual properties information $V_{explo}$ present the provenance traces collected by the \emph{Provenance engine module}.} Captured provenance traces are then processed and analyzed also by this module to extract information interesting for users in the course of the visual data exploration process. 
The processed provenance information is sent ultimately to the \emph{Recommendation engine module} that is described below.



~~\\
\noindent \textbf{Recommendation engine module.} 
As its name indicates, the \emph{Recommendation engine module} generates recommendations to assist users in exploring datasets efficiently. Essentially, it leverages provenance information captured and analyzed by the \emph{Provenance engine module} to produce recommendations.
More specifically, the \emph{Recommendation engine module} computes two types of recommendations:  (i)~recommended queries deemed interesting to the current user, and (ii)~recommended visualizations that are suitable 
%to render sub-sets of manually specified by the user or 
to render the result of a recommended query proposed to the user.


Furthermore, the \emph{Recommendation engine module} assigns an interestingness score $sc$ to each recommended query $Q_{rec}$ to help users in choosing suitable recommendations to study next. 
This incurs ultimately the generation of a set of triples $(Q_{rec},V_{rec},sc)$ that will be rendered through the \emph{Visual user interface module}.










