Visual data exploration allows users to analyze datasets based on visualizations of interesting data characteristics, to possibly discover interesting information about the data.
As the users are a priori unaware of the content of the explored data, the visual data exploration requires an investigative way of navigating through (portions of) the data to discover iteratively valuable information.

Nonetheless, existing visual data exploration tools typically require a tedious process of query and visualization specifications, preventing analysts from efficiently exploring different aspects of the data. 
 As a result, users without enough querying skills or visualization background may fail to accomplish their exploration and overlook important insights or trends hidden in the analyzed data.
 
To address this challenge, this thesis contributed a set of approaches that support users in the visual data exploration process.
Next, we summarize the contributions discussed in this thesis towards improving user 'experience when exploring data visually.
Subsequently, we discuss interesting perspectives and open questions for future investigation.
     
        \section{Review of contributions}
        
In this thesis, we proposed a set of new methods and approaches that aim at improving users' experiences when exploring data visually.

First, we introduced in Chapter~\ref{chap:overview} our framework \framework{} meant to explore visually data. Furthermore, we described thoroughly the components present in our visual data exploration framework. 
Consequently, we introduced \prototype{}, an instance of our framework that is meant to explore visually data warehouses. 

Later, we proposed in Chapter~\ref{chap:evoDM} a new model of evolution provenance that captures prominent information in the course of the visual data exploration process including users' interactions, exploration queries, and visualizations. 
%In this context, we discussed how \prototype{}, the instance of our framework \framework{} collects effectively the evolution provenance in the course of the visual exploration of data warehouses.

After that, %we continued describing the implementation of \prototype{} the instance of our framework \framework{}.
%For that, 
we discussed in Chapter~\ref{chap:EVLIN} the set of provenance-based methods proposed to support users throughout the whole visual data exploration process.
More specifically, we discussed a content-based query recommendation approach that returns the set of interesting queries worth inspecting next.
 To support the visual exploration process, we contributed also a visualization recommendation approach meant to render appropriately investigated recommended queries.
 Indeed, unlike existing visual data exploration that support users either in writing queries or in visualizing data, we proposed these two provenance-based recommendations approaches that assist users in querying and visualizing data. 

Given the high diversity and possibly large number of recommendations produced by our content-based query recommendation approach, we proposed in Chapter~\ref{chap:EVLIN} a quantification approach that measures the ``interestingness'' of each recommendation. The measure of recommendations interestingness relies on the deviation metric that compares the dissimilarity of recommendation's data distribution with the data distribution in the whole explored dataset.
The computed interestingness scores are visualized as an impact matrix, pointing thereby users to potentially interesting recommendations to investigate next.

Furthermore, we described in the same Chapter~\ref{chap:EVLIN} our merge approach that aggregates periodically the evolution provenance collected from many previous users' exploration jobs into a multi-user graph. 
Subsequently, this kind of multi-user graphs is exploited by our second query recommendation approach (described in Chapter~\ref{chap:EVLIN}) that computes collaborative-filtering recommendations.
This new type of recommended queries is harnessed to improve the process of quantifying the interestingness of recommendations output by our content-based query recommendation approach. 
Indeed, we discussed in Chapter~\ref{chap:EVLIN} how we leverage collaborative-filtering recommendation to diversify interestingness scores to 
%reflect global trends (commonality or popularity of such recommendation) as well as local insights.
guide better users to interesting portions of the studied data sets.

All aforementioned approaches underwent an extensive evaluation process that is thoroughly discussed in Chapter~\ref{chap:eval}. Overall, quantitative experiments results showed the efficiency of our provenance-based methods implemented in \prototype{} (our prototype for visual exploration of data warehouses).
We performed also qualitative experiments results to study the effectiveness of our provenance-based methods. The results of qualitative experiments showed general satisfaction among users when visually exploring data using \prototype{}.


Inspired by the important role of evolution provenance aggregation towards supporting collaborative-filtering recommendation in our framework \framework{}, we proposed finally in Chapter~\ref{chap:TaPP19} a new aggregation approach that summarizes provenance documents following W3C-PROV standard. Therefore, we showed the usefulness of our structure-based provenance summaries on several use cases, when using appropriate visualizations and interaction techniques. Additionally, we performed a preliminary experimental evaluation that studies the performance of our proposed provenance summary process. 
Overall, experiments results showed the capability of our proposed approach to process rapidly several input provenance traces and to output concise summaries easy to analyze.
 

	\section{Perspectives}
	  In this section, we briefly present some possible research topics for future work building upon results from this thesis.



\subsection{Interactive and comparative analysis of provenance traces}
\label{sec:future1}
We are interested in researching methods to explore provenance traces, both individually and collectively using interactive visual analytics techniques.
In our current work described in this manuscript, we have the opportunity to collect diverse types of real provenance such as: (i) the evolution provenance traces of data exploration sessions that we generate in our framework \framework{} or that are collected using the visual analytics framework developed in~\cite{Bruder2019} (ii) the why provenance traces collected also in our visual data exploration framework \framework{} when investigating exploration queries' results (iii) the W3C-PROV provenance collected and summarized using our approach discussed in Chapter~\ref{chap:TaPP19}.
We have also implemented diverse synthetic provenance generators that mimic different types of provenance e.g. the evolution provenance (in Chapter~\ref{chap:eval}) and the W3C-PROV provenance traces (in Chapter~\ref{chap:TaPP19}).



Given the availability of various provenance information, an interesting new research direction would be the visual exploration of provenance data. 
Our work described in this thesis shows the prominent role played by visual exploration towards revealing and surfacing insights. Similarly, visual data exploration techniques including namely recommendations could be proposed to analyze provenance information.

The first step towards this goal was already done as described in Chapter~\ref{chap:TaPP19}. Hence, we have proposed a method to summarize a provenance set containing W3C-PROV provenance traces. Accordingly, we have defined a set of visual analysis tasks that could be applied to this kind of summary. 
Performing efficiently the set of defined visual analysis tasks requires the design of convenient visualizations that render appropriately provenance summaries.
As a consequence, we need in the future to propose  a new visualization recommendation approach that renders the suitable visualization of a provenance summary with respect to the visual analysis task.



\subsection{Quantification of uncertain provenance traces}
\label{sec:future2}


The provenance traces collected in this thesis are complete in the sense that they comprise all possible/modeled data of the provenance traces. 
Yet, collecting these complete traces may result in a significant overhead to both the process runtime and storage. 
Given also that we target interactive visual exploration applications, it is important to support quick access to provenance data in the provenance management backend system.
There are already some existing solutions that tackle this issue. For instance, Diestelk{\"a}mper et al.%~\cite{diestelkamper2017provenance}
~\cite{diestelkamper2020} propose an approach that reduces the overhead of provenance computation in DISC systems. 
Overall, these solutions provide interesting approaches that reduce significantly the runtime/storage overhead when computing provenance. Yet, these techniques typically incur some quality issues, e.g., uncertainty etc.

To this end, an interesting new research direction would be to quantify the quality of the provenance information collected by these systems. 
Accordingly, it will be highly interesting to communicate the provenance quality issues to the user as part of the interactive and visual representation of the provenance. To this end, an interesting research avenue will consist of  studying the methods required to quantify and visualize the quality of provenance. 





