
In this thesis, we proposed a set of new methods and approaches that aim at improving users' experiences when exploring data visually.

First, we introduced in Chapter~\ref{chap:overview} our framework \framework{} meant to explore visually data. Furthermore, we described thoroughly the components present in our visual data exploration framework. 
Consequently, we introduced \prototype{}, an instance of our framework that is meant to explore visually data warehouses. 

Later, we proposed in Chapter~\ref{chap:evoDM} a new model of evolution provenance that captures prominent information in the course of the visual data exploration process including users' interactions, exploration queries, and visualizations. 
%In this context, we discussed how \prototype{}, the instance of our framework \framework{} collects effectively the evolution provenance in the course of the visual exploration of data warehouses.

After that, %we continued describing the implementation of \prototype{} the instance of our framework \framework{}.
%For that, 
we discussed in Chapter~\ref{chap:EVLIN} the set of provenance-based methods proposed to support users throughout the whole visual data exploration process.
More specifically, we discussed a content-based query recommendation approach that returns the set of interesting queries worth inspecting next.
 To support the visual exploration process, we contributed also a visualization recommendation approach meant to render appropriately investigated recommended queries.
 Indeed, unlike existing visual data exploration that support users either in writing queries or in visualizing data, we proposed these two provenance-based recommendations approaches that assist users in querying and visualizing data. 

Given the high diversity and possibly large number of recommendations produced by our content-based query recommendation approach, we proposed in Chapter~\ref{chap:EVLIN} a quantification approach that measures the ``interestingness'' of each recommendation. The measure of recommendations interestingness relies on the deviation metric that compares the dissimilarity of recommendation's data distribution with the data distribution in the whole explored dataset.
The computed interestingness scores are visualized as an impact matrix, pointing thereby users to potentially interesting recommendations to investigate next.

Furthermore, we described in the same Chapter~\ref{chap:EVLIN} our merge approach that aggregates periodically the evolution provenance collected from many previous users' exploration jobs into a multi-user graph. 
Subsequently, this kind of multi-user graphs is exploited by our second query recommendation approach (described in Chapter~\ref{chap:EVLIN}) that computes collaborative-filtering recommendations.
This new type of recommended queries is harnessed to improve the process of quantifying the interestingness of recommendations output by our content-based query recommendation approach. 
Indeed, we discussed in Chapter~\ref{chap:EVLIN} how we leverage collaborative-filtering recommendation to diversify interestingness scores to 
%reflect global trends (commonality or popularity of such recommendation) as well as local insights.
guide better users to interesting portions of the studied data sets.

All aforementioned approaches underwent an extensive evaluation process that is thoroughly discussed in Chapter~\ref{chap:eval}. Overall, quantitative experiments results showed the efficiency of our provenance-based methods implemented in \prototype{} (our prototype for visual exploration of data warehouses).
We performed also qualitative experiments results to study the effectiveness of our provenance-based methods. The results of qualitative experiments showed general satisfaction among users when visually exploring data using \prototype{}.


Inspired by the important role of evolution provenance aggregation towards supporting collaborative-filtering recommendation in our framework \framework{}, we proposed finally in Chapter~\ref{chap:TaPP19} a new aggregation approach that summarizes provenance documents following W3C-PROV standard. Therefore, we showed the usefulness of our structure-based provenance summaries on several use cases, when using appropriate visualizations and interaction techniques. Additionally, we performed a preliminary experimental evaluation that studies the performance of our proposed provenance summary process. 
Overall, experiments results showed the capability of our proposed approach to process rapidly several input provenance traces and to output concise summaries easy to analyze.
 
