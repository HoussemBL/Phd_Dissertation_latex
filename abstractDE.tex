Visuelle Datenexploration hilft Nutzern, interessante Informationen in Datens\"atzen zu identifizieren, die ihnen weitgehend unbekannt sind. Damit werden Nutzer beim Verst\"andnis und bei der Interpretation von Daten auf investigative Weise unterst\"utzt. Typischerweise gehen Systeme zur visuellen Datenexploration davon aus, dass deren Nutzer ausreichend qualifiziert sind, um explorative Anfragen zu formulieren und geeignete Visualisierungen f\"ur die Anfrageergebnisse zu konstruieren. Dieser manuelle Prozess ist oftmals zeitaufw\"andig. In realen Szenarien, in denen Nutzer oft nur wenig Zeit f\"ur die visuelle Erforschung gro\ss{}er Datens\"atze haben, ist dies nicht zweckm\"a\ss{}ig.

%Um die Benutzer bei der effizienten Durchf\"uhrung der visuellen Datenexploration zu unterst\"utzen, stellt diese Di\ss{}ertation das Design von Systemen zur visuellen Datenexploration vor, die den Benutzer im gesamten Explorationsproze\ss{} unterst\"utzen.
Um die Effizienz visueller Datenexploration zu verbessern, stellt diese Dissertation ein Rahmenwerk für interaktive visuelle Datenexploration vor. Es verfolgt einen ganzheitlichen Ansatz zur Unterst\"utzung der Nutzer im gesamten Explorationsprozess, indem diesen Empfehlungen für Anfragen und deren Ergebnisvisualisierung unterbreitet werden.
Hierbei setzen wir auf die Verwertung sogenannter Provenance. Im Allgemeinen versteht man unter Provenance Metadaten, die den Prozess zur Erlangung von Daten beschreiben. In unserem Kontext wird Provenance erfasst, die die verschiedenen Phasen des visuellen Datenexplorationsprozesses dokumentiert. Diese Provenance wird verwendet, um Empfehlungen f\"ur Nutzer zu generieren. %im gesamten Prozess der visuellen Datenexploration durch Empfehlungen zu unterst\"utzen.


%Die gesammelte Provenanceinformationen sowie andere Provenancetypen werden in dieser Arbeit genutzt, um die Benutzer im gesamten Prozess der visuellen Datenexploration zu unterst\"utzen.

Die wissenschaftlichen Beitr\"age dieser Dissertation lassen sich wie folgt zusammenfassen. Zun\"achst schlagen wir 
ein neues \emph{Evolution-Provenance} Modell vor, das alle wichtigen Aspekte im Zusammenhang mit dem visuellen Datenexplorationsprozess erfasst, wie z.B. die Anfragen, die ein Nutzer im laufenden Prozess bereits analysiert hat, Interaktionen der Nutzer und visuelle Kodierungsparameter der entsprechenden Visualisierungen.

Auf der Grundlage dieses Modells und einer anderen Art von Provenance schlagen wir verschiedene neuartige \emph{Empfehlungsans\"atze} vor, die den Anwendern bei der Abfrage und Visualisierung von Daten helfen. Diese Ans\"atze erzeugen verschiedene Arten von Empfehlungen, die den gesamten Datenraum des untersuchten Datensatzes abdecken und die es den Nutzern erm\"oglichen, leicht gerenderte Ergebnisse einzusehen.

Unter Verwendung unserer Empfehlungsans\"atze erhalten Nutzer in den verschiedenen Phasen des visuellen Datenexplorationsprozesses mehrere Alternativen empfohlen. Dementsprechend steuern wir einen \emph{Ansatz zur Quantifizierung der Interessantheit} einer Empfehlung bei. Der Grundgedanke dahinter ist, den Nutzer dazu anzuleiten, die Empfehlungen auszuw\"ahlen, die es wert sind, als n\"achstes untersucht zu werden.

All diese neuartigen Ans\"atze werden durch unser auf Provenance basierendes Framework integriert, das einen ganzheitlichen Ansatz zur Unterst\"utzung der Nutzer in allen Phasen des visuellen Datenexplorationsprozesses bietet. Eine quantitative und qualitative Evaluation belegt, dass unsere Beitr\"age den Stand der Forschung verbessern und Nutzer effektiv bei der visuellen Datenexploration unterst\"utzen.

W\"ahrend die bisher genannten Beitr\"age auf die Verwendung von Provenance f\"ur visuelle Datenexploration fokusieren, stellten wir w\"ahrend unserer Forschung fest, dass die Analyse einer Menge von Provenance Datens\"atzen eine allgemeinere Forschungsfrage darstellt. Aus diesem Grund schlagen wir einen Ansatz der \emph{Provenance-Aggregation} vor, der mehrere Provenance Datens\"atze zusammenfasst und er\"ortern m\"ogliche visuelle Analyseaufgaben, die auf diese Art von Zusammenfassung angewendet werden k\"onnen.

\smallskip 
\noindent \textbf{Schlagw\"orter: }Visuelle Datenexploration, Provenance, Empfehlungsalgorithmen, Provenance-Aggregation

%
%Um den Anwendern zu helfen, die visuelle Datenexploration effizient durchzuf\"uhren, stellt diese Di\ss{}ertation das Design von visuellen Datenexploration\ss{}ystemen vor, die den Anwender bei seiner Analyse unterst\"utzen, indem sie neben der Analyse Empfehlungen vorschlagen.
%Unsere Forschung nutzt Herkunftsdaten, die aus der Erfa\ss{}ung von Informationen bestehen, die den verfolgten Proze\ss{} oder die Herkunft der Daten beschreiben. Diese besondere Art von Daten wurde in~\cite{Herschel2017survey} als eine intere\ss{}ante Information dargestellt, die in verschiedenen Anwendungen n\"utzlich sein kann.  Dementsprechend zielen wir in dieser Arbeit darauf ab, die Herkunftsinformationen zu sammeln, um verschiedene Phasen des visuellen Datenexplorationsproze\ss{}es zu dokumentieren und zu kommentieren. 
%Die gesammelten Datenherkunft sowie andere Herkunftsorten werden in dieser Arbeit genutzt, um verschiedene Formen der Empfehlung anzubieten. 
%Wir geben zun\"achst einen \"uberblick \"uber verschiedene Konzepte und Paradigmen, die notwendig sind, um unsere Beitr\"age zu erfa\ss{}en.  
%
%Danach stellen wir ein neues \emph{Evolutionsherkunft} Modell vor, das wir vorschlagen, alle wichtigen Aspekte im Zusammenhang mit dem Proze\ss{} der visuellen Datenexploration zu erfa\ss{}en, wie z.B. die Explorationsabfragen der Benutzer, die Interaktionen der Benutzer und die visuellen Kodierungsparameter der entsprechenden Visualisierungen.
%
%Basierend auf diesem Modell und einer anderen Art von Herkunft schlagen wir zwei Empfehlungsans\"atze vor, die Anwendern helfen, Daten effizient zu erforschen. Der erste vorgeschlagene Ansatz empfiehlt Explorationsabfragen, die f\"ur die Nutzer potenziell intere\ss{}ant sind, w\"ahrend der zweite vorgeschlagene Ansatz geeignete Visualisierungen empfiehlt, die die Daten klar darstellen und damit die Interpretation der Ergebni\ss{}e erleichtern.
%Angesichts der gro\ss{}en Anzahl von Empfehlungen, die von unserem Query-Recommendation-Ansatz ausgegeben werden, tragen wir zu einem Quantifizierungsansatz bei, der jeder Empfehlung einen Zinscore zuordnet. Diese Ergebni\ss{}e werden den Benutzern zur Verf\"ugung gestellt, um hochintere\ss{}ante Empfehlungen hervorzuheben.
%Um die Benutzer bei der Pr\"ufung des Empfehlung\ss{}atzes be\ss{}er zu unterst\"utzen, schlagen wir einen zweiten Ansatz f\"ur die Abfrageempfehlung vor, der der Richtlinie f\"ur kollaboratives Filtern folgt. 
%Letzteres nutzt eine zusammengef\"uhrte Version der Evolutionsherkunft Daten (repr\"asentativ f\"ur die Erkundungen fr\"herer Benutzer), um Erkundungen zu f\"ordern, die h\"aufig zuvor von den Benutzern gew\"ahlt wurden. Genauer gesagt, wird der kollaborativ-filternde Empfehlungsansatz in unserer Arbeit verwendet, um den Quantifizierungsproze\ss{} der Empfehlung zu verst\"arken , indem global intere\ss{}ante Trends zus\"atzlich zu den (begrenzten) lokalen, die mit dem inhaltsbasierten Empfehlungsansatz gewonnen wurden, ber\"ucksichtigt werden. Alle unsere vorgeschlagenen Ans\"atze werden in unserem Framework f\"ur die visuelle Datenexploration implementiert und gesammelt, das f\"ur die Untersuchung von Daten gedacht ist.
%
%
%Schlie\\ss{}{}lich, inspiriert von der entscheidenden Rolle, die die Aggregation der Datenherkunft (Merge der Evolution Provenance) bei der Unterst\"utzung der kollaborativen Filterung der Query-Empfehlung spielt, haben wir einen generischeren Datenherkunft-Aggregationsansatz vorgeschlagen, der Herkunftspuren zusammenfa\ss{}t, die dem W3C-PROV Standard folgen. Dementsprechend diskutieren wir unsere L\"osung f\"ur die summarische Datenherkunft sowie m\"ogliche visuelle Analyseaufgaben, die auf diese Art von Zusammenfa\ss{}ung angewendet werden k\"onnen.
%
%Schl\"u\ss{}elw\"orter: Visuelle Datenexploration, Evolutionsherkunft, Datenherkunft, Empfehlung, Datenherkunft aggregation