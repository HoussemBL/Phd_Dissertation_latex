\section{Introduction}
\label{sec:intro}

So far, we have described the evolution provenance model supported in our framework. In this chapter, we discuss the \emph{recommendation engine} module present in the architecture of our framework (cf.~Figure~\ref{fig:archi-FW}).
To do that, we specialize to a specific scenario, i.e., data warehouse exploration to leverage clear semantics of typical recommendations that can be suggested when exploring data warehouses. 
More specifically, we discuss the following novel provenance-based contributions implemented in \prototype{} (the instance of our framework \framework{} that specializes to the visual exploration of data warehouses).

\begin{itemize}

\item We propose a \emph{content-based query recommendation algorithm} that leverages data provenance to guide users to different portions of the studied data sets. 



\item  We propose a method to quantify recommendation quality.  Given the high diversity and possibly large number of recommended queries 
produced by our content-based query recommendation approach, we propose to support users to navigate through the exploration space by quantifying the ``interestingness'' of recommended queries. The computed scores are then visualized in an interactive impact matrix, pointing users to potentially interesting recommendations to study next.


\item We discuss several techniques to incrementally merge evolution provenance graphs, which
 describe individual user exploration sessions, into a global graph. The different approaches trade off merge efficiency and effectiveness. The proposed approaches are meant to analyze global trends adopted by users when exploring visually the same data.
 

\item We propose a \emph{collaborative-filtering method} for query recommendation that leverages the merged graph (discussed in the previous point) to promote recommendations widely inspected previously by users. 
Our collaborative-filtering query recommendation approach is blended with our content-based query recommendation approach to improve the quantification process of recommended queries by taking into account globally interesting trends in addition to (limited) local information.

\item 
We outline a method that recommends visualizations for a query result, once a user has selected a (recommended) query. The goal of these recommendations is to provide visualizations that render the data appropriately and facilitate thereby the interpretation of results.  This increases potentially the understandability and thereby the efficiency of visual data exploration.


\end{itemize}



In what follows, we discuss in Section~\ref{sec:content-query-rec}, our content-based query recommendation approach. Section~\ref{quantification-rec} covers our approach proposed to quantify the interestingness of recommended queries. 
Then, we discuss in Section~\ref{collaborative-query-rec} our collaborative-filtering query recommendation approach that aggregates evolution provenance and uses it to promote users' global trends.
%We describe also in the same Section~\ref{collaborative-query-rec} how information output by our collaborative-filtering recommendation approach is harnessed to improve the recommendation quantification process discussed in Section~\ref{quantification-rec}.
%{\color{Fuchsia}
Furthermore, we describe also in the same Section~\ref{collaborative-query-rec} how we leverage the collaborative-filtering recommendation approach to improve the recommendation quantification process discussed in Section~\ref{quantification-rec} by providing.
Subsequently, we discuss in Section~\ref{sec:vis-rec} our visualization recommendation approach. %proposed to render appropriately recommended queries' results. 
Alongside the description available in each aforementioned section, we review important existing work related to each contribution.

Note that the content of this chapter is based on methods and approaches described in~\cite{Houssem:17:tapp,BHBK18:Demo,Houssem:19:adbis,Houssem:19:IS}.


