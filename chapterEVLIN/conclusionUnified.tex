 \section{Conclusion}
 
% \mel{Do not make your contribution about the implementation of a part of EVLIN. Be more general. Recommendation engine that integrates both query and visual recommendations based on provenance. This fits your general framework. Only in a second sentence you say that the techniques are tailored to exploration of data warehouses, leveraging specificities of this application for specific optimizations necessary for good quality results with interactivity requirement. Finally, say that all techniques are implemented in your system prototype EVLIN, which is the basis for the evaluation discussed next. \\As this chapter is very dense, you should summarize the contributions. So after general summary (see above), you list again provenance-based content-based query recommendation; 2 graph merging strategies for collaborative filtering based on evolution provenance, X variants of online query recommendation using merged graph, novel metric for vis recommendation.}
% 
 

%In this chapter, we presented our scientific contributions proposed to implement the \emph{recommendation engine} module in the system \prototype{}, the instance framework \framework{}.
In this chapter, we presented our scientific contributions proposed to implement the \emph{recommendation engine} module of our visual data exploration framework \framework{}.

In particular, we discussed our content-based query recommendation meant to assist users in exploring interesting data regions.
Furthermore, we described our approach proposed to quantify the interestingness of recommended queries

Second, we investigated our merge methods meant to fuse evolution provenance graphs, our collaborative-filtering query recommendation method and how this new recommendation approach is employed to reinforce the process of quantifying the interestingness of recommendations.

Finally, we discussed our visualization recommendation showing thereby how our framework \framework{} integrates seamlessly both query and visualization recommendation for an interactive visual exploration user-experience.


Finally, we point out that these techniques are tailored to the exploration of data warehouses, leveraging specificities of this application for specific optimizations necessary for good quality results with interactivity requirement. Therefore, all these techniques are implemented in our system \prototype{}, which is the basis for the evaluation discussed next.




