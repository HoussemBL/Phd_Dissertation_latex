This chapter provides an overview of the different concepts and paradigms that are necessary to define and put into perspective our contributions. 
We start by reviewing the visual data exploration process, its main steps, and pertinent existing work in Section~\ref{sec:visexplorprocess}.
%As we present in this thesis a sample of the visual exploration of data warehouses,  
As the recommendation methods we contribute in this thesis are specialized for visual data exploration in data warehouses, we discuss basic concepts of data warehouses in Section~\ref{sec:olap}.
Finally, given that our contributions leverage provenance data, we dedicate the rest of this chapter to defining the provenance and its types. %, its applications and its requirements (\mel{Again I think these are not necessary}) in Section~\ref{prov:def}. 
Note that Section~\ref{prov:def} is mainly based on classifications, definitions, and descriptions published in our survey~\cite{Herschel2017survey}.
