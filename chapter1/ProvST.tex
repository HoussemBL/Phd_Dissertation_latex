
We have shown in the previous section the various types of provenance.
This diversity of provenance types impacts significantly the content and the structure of provenance. 
Indeed, there are a myriad of provenance solutions collecting diverse provenance types with various formats.
%The variation of these output formats is very high. For this amount of different output formats using a provenance standard as a unified format is proposed. 
%This complicates the communication of provenance information itself. 
Using a provenance standard has advantages when combining many provenance systems to compare between their results or to get a full story about the production of such result. To this end, two provenance standards were proposed. The first one is the open provenance model (OPM) [OPM]. The second standardization attempt results in the provenance standard developed by World Wide Web Consortium (W3C) [W3Ca].

We focus on what follows  on W3C-PROV as it is the most recent standard and given that it is more adopted. 


W3C-PROV standard consists of a family of documents. All these documents are based on the same main components which are discussed later. Some of these documents are PROV-XML [W3Cg], PROV-O [W3Ce], PROV-JSON [W3Cf], PROV-N [W3Cd], etc. . W3C-PROV has the main advantages that it enables interoperability and interchange of provenance on the web [W3Ca]. As we will make use of PROV-N and PROV-JSON in this thesis, we will explain them in detail later.
The main components [W3Cc] of the W3C-PROV standard are shown in Figure 2.2. The three core types are entities, activities and agents. Between these core types connections are existing. These connections are called relations. Relations are taking place between one, two or sometimes even three core types.
