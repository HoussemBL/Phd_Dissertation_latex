



In this chapter, we performed a series of evaluations to validate our contributions implementing the \emph{provenance engine} and the \emph{recommendation engine} modules of our framework \framework{} (cf.~Figure~\ref{fig:archi-FW}) in our system \prototype{}.
More specifically, we evaluated first our content-based query recommendation approach and our methods quantifying recommended queries.  Second, we investigated our merge methods meant to fuse evolution provenance graphs, our collaborative-filtering query recommendation method and its impact on improving the quantification of recommendation.
Finally, we evaluated our visualization recommendation approach.

Overall, these contributions implemented in our system \prototype{}, were evaluated  both quantitatively using performance measurements and qualitatively with a user study on both synthetic and real data. 
Quantitative experiments show the feasibility and the efficiency of our proposed solutions for visual and interactive data exploration while qualitative evaluations show a general satisfaction among users when visually exploring data warehouses data using our system \prototype{}. 



While our contributions evaluated in this chapter focus mainly on using provenance for visual data exploration, we noticed during our research, in particular for our collaborative-filtering recommendation approach, that there is a more general research question of analyzing a set of provenance documents. To better support such analysis, we discuss in the next chapter, a general provenance summary approach that processes diverse types of provenance (including evolution provenance) and it aims at facilitating users' task when exploring visually provenance.



