\label{sec:evo-instances}
Our evolution provenance data model is adopted in the two instances of our framework {\color{Fuchsia}\framework{}}. 
The first instance of our framework is \prototype{}~\cite{BHBK18:Demo} that offers a visual exploration of data warehouses.
The second instance of our framework is the volume-based large dynamic graph analysis solution~\cite{Bruder2019} that offers an interactive visual exploration of large dynamic graphs.


Note that, alongside the description of the evolution data model in Section~\ref{sec:evo-core}, we provided several examples (e.g., Examples~\ref{ex:query-path},~\ref{ex:expo-session}, and~\ref{ex:multi-userG}) showing the feasibility of evolution provenance computation in \prototype{}.


Therefore, we dedicate the rest of this section to discuss the incorporation of our evolution provenance data model into the second instance of our framework that allows for interactive analysis of dynamic graphs. 



\subsection{Overview of large dynamic graph exploration}\label{A02:overview}
%Bruder et al.~\cite{bruder:18} propose recently a visual analytics approach that allows for interactive analysis of dynamic graphs containing several thousand time steps and hundreds of nodes.
%	
%
%Inspired by space-time cube approaches e.g.,~\cite{bach:2017} and~\cite{Bach_CHI:14}, the proposed approach adopts a volumetric representation of the dynamic graph based on its adjacency matrices.
%These concepts have in common that they stack representations of individual time steps to gain a three dimensional data structure.
%As illustrated in Figure~\ref{fig:illu}, adjacency matrices are 2D structures that are stacked to incorporate the temporal evolution of the graph.
%In the resulting three dimensional cuboid structure, the $x$- and $y$-axes represent nodes, while entries in the plane defined by those two axes represent edges (including their weights). 
%The $z$-axis represents time.
%
%\begin{figure}
%	\centering
%	\includegraphics[width=1.0\linewidth]{figures/evoDM/stacking-crop}
%	\caption{Sample of explored adjacency matrices~\cite{Bruder2019}}
%	\label{fig:illu}
%\end{figure}
%
%
%
%This cuboid is a concise and static representation of the whole dynamic graph that preserves the mental map~\cite{Misue:95,Archambault:11}. 
%To allow for fluent user interactions (e.g., rotating and zooming into the cuboid or filtering values), GPU-accelerated volume rendering methods are employed to generate the graph data visualizations. 
%Using those techniques, the proposed approach~\cite{bruder:18} is capable to render several thousand time steps and hundreds of nodes at the same time without loosing interactivity.
%
%
%This approach offers a set of analytic methods, used to perform typical analysis tasks such as detecting temporal patterns, e.g., clusters forming or repeating occurrences of similar node-link structures.
%Essentially, we distinguish three classes of analytic methods.
%
%\begin{description}
%	\item[Data Views.] 
%	Large dynamic graphs typically exhibit a lot of information covering different aspects of interest. 
%	A single visualization is often not capable to convey them all.
%	Therefore, it is important to offer distinct perspectives on the data using multiple views with suitable visualizations. 
%	The volume view is the primary visualization adopted by this approach.  Indeed, it provides a direct visualization of the full graph volume. This latter could be split into sub-volumes along the time axis. This is possible using the volume partitioning perspective.
%	Other views include for instance the timeline plot used to show different graph metrics on a 2D plot over time. 
%	This approach offers also detailed visualization such as Slice view that provides information of individual, selected time steps with 2D slice  views. 
%	\item[Aggregation and Filtering.] 
%	Showing all edges of a large dynamic graph, especially a dense one, quickly leads to visual clutter, overload, and occlusion using our volumetric approach. 
%	This makes filtering and aggregation of adjacent edges an essential part of the analysis process to reduce the visualization to relevant information.
%	Examples of aggregation functions include for instance minimum, maximum, average weight and density of edges. These methods could be applied either on in the spacial domain i.e. aggregating neighboring edges or in the time domain.
%	Filtering methods consists of omitting edges not satisfying such predicate e.g. filter out edges with low density.
%	\item[Comparison.] 
%	Comparing different sections within a temporal graph or even several distinct dynamic graphs becomes challenging using the methods discussed above. 
%	The support of a dedicated visualization for this specific task is therefore important for the analysis process.
%	This is possible using this approach. Indeed, an analyst can select starting points and a range of time steps to compare them against one another.
%	In this case, a matrix view is generated to render commonality and irregularities resulting from comparing the different time sequences.
%\end{description}

The volume-based large dynamic graph analysis solution is an instance of our visual data exploration framework {\color{Fuchsia}\framework{}}.
It is based on a volume-based approach proposed in~\cite{Bruder2019} where we propose methods to allow for interactive analysis of dynamic graphs containing several thousand time steps and hundreds of nodes.
	

%Inspired by space-time cube approaches e.g.,~\cite{bach:2017} and~\cite{Bach_CHI:14}, authors in~\cite{Bruder2019} adopt a volumetric representation of the dynamic graph based on its adjacency matrices.
%These concepts have in common that they stack representations of individual time steps to gain a three dimensional data structure.
%As illustrated in Figure~\ref{fig:illu}, adjacency matrices are 2D structures that are stacked to incorporate the temporal evolution of the graph.
%In the resulting three dimensional cuboid structure, the $x$- and $y$-axes represent nodes, while entries in the plane defined by those two axes represent edges (including their weights). 
%The $z$-axis represents time.
Inspired by space-time cube approaches (e.g.,~\cite{bach:2017} and~\cite{Bach_CHI:14}), the instance of our framework adopts a volumetric representation of the dynamic graph.
As illustrated in Figure~\ref{fig:illu}, individual time steps of the dynamic graph are initially represented as adjacency matrices.
Those latter are stacked into space-time cubes.
This results in three dimensional volume structure where the $x$- and $y$-axes represent nodes, while entries in the plane defined by those two axes represent edges (including their weights). 
The $z$-axis represents time.

\begin{figure}
	\centering
	\includegraphics[scale=0.3]{figures/evoDM/stacking-crop}
	\caption{Sample of explored adjacency matrices~\cite{Bruder2019}}
	\label{fig:illu}
\end{figure}


To allow for fluent user interactions, the instance of our framework~\cite{Bruder2019} offers a set of analytic methods, used to explore large dynamic graphs.
Essentially, we distinguish three classes of analytic methods {\color{Fuchsia}(equivalent to exploration queries in our framework)}.

\begin{description}
	\item[Data Views.] 
A single visualization is often not capable to convey all information of the exhibited large dynamic graphs.
Therefore, several data views are proposed to visualize large dynamic graphs.
This includes for instance the volume view, the timeline plot and slice views that are detailed in~\cite{Bruder2019}.
	\item[Aggregation and Filtering.] 
Filtering and aggregation of adjacent edges an essential part of the analysis process to reduce the visualization of large dynamic graphs to relevant information.
Examples of aggregation functions include for instance minimum, maximum, average weight and density of edges. These methods could be applied either on in the spacial domain i.e. aggregating neighboring edges or in the time domain.
Filtering methods consists of omitting edges not satisfying such predicate, e.g., filter out edges with low density.
\item[Comparison.] 
Comparing different sections within a temporal graph or even several distinct dynamic graphs is supported via the comparison function. 
In this case, a matrix view is generated to render commonality and irregularities resulting from comparing the different time sequences or distinct dynamic graphs.
\end{description}

		
\subsection{Evolution provenance model}\label{A02:evoDM}
Using our volume-based large dynamic graph analysis solution~\cite{Bruder2019},
users are engaged in a \emph{visual analysis session} where they perform various \emph{visual analysis steps} iteratively. 
In this case, evolution provenance keeps track of the set of visual analysis steps performed and thereby construct the ``full story'' of a visual analysis session.
For that, we have implemented our evolution provenance model presented in Section~\ref{subsec:evo} to track the visual analytics process of large dynamic graphs. 
%Using the volume-based large dynamic graph analysis solution~\cite{Bruder2019},
%users are engaged in a \emph{exploration session} where they perform various \emph{exploration steps} iteratively. 
%In this case, evolution provenance keeps track of the set of exploration steps performed and thereby construct the ``full story'' of a exploration session.
%For that, we have adjusted slightly our evolution provenance model presented in Section~\ref{subsec:evo} to track the visual analytics process of large dynamic graphs. 



{\color{Fuchsia}In what follows, we detail how the volume-based large dynamic graph analysis solution~\cite{Bruder2019} incorporates the evolution provenance data model presented in Section~\ref{sec:evo-core}. To do that, we define necessary analogous concepts necessary to define the evolution provenance.}


\begin{definition}[Visual analysis step] 
Given an initial dynamic graph $G_i$ that contains a set of timesteps $\{t_1..t_i\}$, we define a visual analysis step as $\exploreStep$ where a dynamic graph $G_s$  that contains $\{t_i..t_j\}$ timesteps  (where $1 \leq i \leq j \leq k $) is visualized using a set of views $V_s=\{v_1..v_p\}$.
\end{definition}


Notice that view is a generic term that covers visualization structures possibly obtained using \emph{data views} analytic method.
In our current implementation, the evolution provenance collector tracks visual analysis steps encompassing the volumetric representation as our main view.
For future work, we also plan to integrate tracking of slice views and the timeline plot.

%\begin{table}[t]
%\taburowcolors[2]{white .. black!10}
%\sffamily\footnotesize
%\tabulinesep=4pt
%\begin{tabu}{|X[cm]|X[cm]|X[cm]|}
%\hline
%\rowcolor{black!80} \color{white}Operation type &   \color{white}Parameters& \color{white}Output\\
%Selection&  $\{t_{i'}..t_{j'}\}$; selected range of timesteps & $G'_{s}=\{t_{i'}..t_{j'}\}$,  $V'_{s}=\{v'_{1}..v'_{p}\}$\\
%Partition&$\{m_1..m_y\}$; the set of split marks & $G'_{s}=[\{t_1..t_m\},\{t_{m+1}..t_l\},..]$ ,  $V'_{s}=\{v'_{1}..v'_{p}\}$ \\
%Aggregation&    $l$, where $l$ is a level & $G_{s}$,  $V'_{s}=\{v'_{1}..v'_{p}\}$ \\
%Filtering&    $cond$, where $cond$ is a predicate& $G_{s}$,  $V'_{s}=\{v'_{1}..v'_{p}\}$ \\
%Color mapping&  $d \times rgb $, where $rgb$ is color and $d$ a graph property &$G_{s}$,  $V'_{s}=\{v'_{1}..v'_{p}\}$ \\
%Camera configuration&  $configuration$   & $G_{s}$, $V'_{s}=\{v'_{1}..v'_{p}\}$ \\
%\hline
%\end{tabu}
%\caption{Permitted analytics operations~\cite{Bruder2019}}
%\label{table:ops}
%\end{table}



\begin{table}[t]
 \centering \scriptsize
 \begin{tabular}{|p{2.5cm}|p{4cm}|p{3cm}|} \hline
\textbf{Operation type}  & \textbf{Parameters} & \textbf{Output}  \\ \hline
Selection&  $\{t_{i'}..t_{j'}\}$; selected range of timesteps & $G'_{s}=\{t_{i'}..t_{j'}\}$,  $V'_{s}=\{v'_{1}..v'_{p}\}$ \\ \hline
Partition&$\{m_1..m_y\}$; the set of split marks & $G'_{s}=[\{t_1..t_m\},\{t_{m+1}..t_l\},..]$ ,  $V'_{s}=\{v'_{1}..v'_{p}\}$  \\ \hline
Aggregation&    $l$, where $l$ is a level & $G_{s}$,  $V'_{s}=\{v'_{1}..v'_{p}\}$ \\ \hline
Filtering&    $cond$, where $cond$ is a predicate& $G_{s}$,  $V'_{s}=\{v'_{1}..v'_{p}\}$  \\ \hline
Color mapping&  $d \times rgb $, where $rgb$ is color and $d$ a graph property &$G_{s}$,  $V'_{s}=\{v'_{1}..v'_{p}\}$  \\ \hline
Camera configuration&  $configuration$   & $G_{s}$, $V'_{s}=\{v'_{1}..v'_{p}\}$  \\ \hline
\end{tabular}
\caption{Permitted analytics operations~\cite{Bruder2019}}
\label{table:ops}
 \end{table}

Table~\ref{table:ops} summarizes supported analytical operations enabling the navigation from a visual analysis step $\exploreStep$ to another step $\exploreStepDest$.
It presents also the set of parameters needed for each operation as well as the structure of the output analysis step $S'$.
Essentially, we distinguish six operation types that handle various views seen over several analysis steps.
All proposed operations to track in evolution provenance are derived from the three fundamental analytic methods thoroughly discussed in Section~\ref{A02:overview} that are data views, filtering/aggregation and comparison.


The \emph{selection function} is associated to the timeline plot feature where the user gets a 2D-visualization whose x-axis depicts a set of timesteps $\{t_i..t_j\}$.
Using this function, the user can select a single or a range of timesteps $\{t_{i'}..t_{j'}\}$  with $t_i \leq t_{i'} \leq t_{j'} \leq t_{j}$ to analyze them visually in the next step.

The \emph{partition operation} corresponds to the volume partitioning analysis feature 
where the user specifies interactively some split marks $\{m_1..m_y\}$ to split the timesteps $\{t_1..t_i\}$ associated to a visual analysis step $S$ into sub-ranges $[\{t_1..t_m\},\{t_{m+1}..t_l\},\ldots]$. 


The evolution provenance model implemented for the volume-based large dynamic graph analysis solution encompasses also the \emph{aggregation operation} and the \emph{filtering operation} (via opacity). 
The former aggregates to a specific level $l$ to alleviate the complexity of a view $V$, while the latter operation omits information available in the current view that do not satisfy a predicate $cond$. As shown in Table~\ref{table:ops}, the two aforementioned operations introduce changes only on the set of views to analyze in the step $S'$ in comparison to step $S$ while keeping the same dynamic graph $G_s$.

The list of analytics operations recorded by our provenance model also contains the \emph{color mapping operation} where a user maps a graph property $d$ (e.g., weights of edges) to a specific range of colors $rgb$ to produce new views $V'_{s}=\{v_{i'}..v_{j'}\}$ for the same dynamic graph $G_s$, seen in the analysis step $S$. 
Finally, we record selected \emph{camera configurations}, where the user selects a certain zoom level, rotation and panning of the camera to get new views $V'_s$ in the next analysis step $S'$.

%Overall, the evolution provenance is modelled by an analysis session graph that gathers all visual analysis steps made by the analyst. 
%Figure~\ref{fig:expo-session} shows an example of such an analysis session graph (augmented with exemplary screenshots of the analytics step), defined as follows.
%
%\begin{figure}[t]
%	\begin{center}
%	\includegraphics[scale=0.35]{figures/evoDM/sessionGraph-crop}
%			\end{center}
%			\caption{Example of an analysis session graph, augmented with images of the respective analytics steps~\cite{Bruder2019}}
%			\label{fig:expo-session}
%\end{figure}
%
%\begin{definition}[Analysis session graph]
%An analysis session graph summarizes user's manipulations over a large Dynamic Graph $D_i$ where $i$ refers to the set of timesteps.
%The analysis session graph is a labeled directed acyclic graph (DAG) $\sessionGraph{}_{D_i}(\sessionV{}, \sessionE{})$ where $\sessionV{}$ is a set of nodes and $\sessionE{}$ a set of labeled edges. 
%Each node $n \in \sessionV$ corresponds to a visual analytics step $S$.
%An edge $e = (n, n', \sessionL{})$ represents the transition from one visual analytics step $S = \{{G_s}, V_s\}$ to the next visual analytics step $S' = \{{G'_s}, V'_s\}$. $\sessionL{}$ is a pair $\langle \labelOp{}, param \rangle$ where $\labelOp{}$ is an identifier of the analytical operation type (see Table~\ref{table:ops}) and $param$ is the set of parameters used to navigate from $S$ to $S'$.
%\label{def:sessionA02}
%\end{definition}

\begin{figure}[t]
	\center
	\includegraphics[scale=0.35]{figures/evoDM/sessionGraph-crop}
			\caption{Example of an analysis session graph, augmented with images of the respective analytics steps~\cite{Bruder2019}}
			\label{fig:expo-session}
\end{figure}
{\color{Fuchsia}Overall, the evolution provenance is implemented in the volume-based large dynamic graph analysis solution~\cite{Bruder2019} following our proposed data model specified in Definition~\ref{def:session}.
Accordingly, the evolution provenance of the volume-based large dynamic graph analysis solution~\cite{Bruder2019} is a graph that gathers all visual analysis steps (analogous to exploration steps) made by the analyst. 
Figure~\ref{fig:expo-session} shows an example of such an analysis session graph (augmented with exemplary screenshots of the analytics step). }%, defined as follows.

Essentially, nodes of the evolution provenance graph correspond to the set of visual analytics steps and edges represents the transition from one visual analytics step to another. %visual analytics step. 
Note that, the labels of edges contain so far only type of navigation and set of parameters used to navigate. In the future, we intend to implement the score $s(e)$ (specified in Definition~\ref{def:session}) that reflects the importance of each edge present in the evolution provenance graph.
