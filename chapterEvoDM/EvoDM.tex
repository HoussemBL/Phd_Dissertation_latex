\section{Introduction}
\sloppy


We have seen in Chapter~\ref{chap:overview} that our framework \framework{} relies on provenance, i.e.,~meta-data about a user's exploration process to determine and quantify recommended queries and visualizations. This chapter now introduces the novel provenance model that allows us to provide such interrelated recommendations.   
This relies mainly on formalizations and definitions elaborated in our work~\cite{Houssem:17:tapp,Houssem:19:adbis}.

For that, we discuss first in Section~\ref{sec:evo-DM-related} existing evolution provenance data models. Then, we introduce in Section~\ref{sec:evo-core} the evolution provenance data model adopted in our framework \framework{}.  
%Finally, we conclude this chapter before introducing for the next chapter.

\section{State of the art of evolution provenance}
%OLD VERSION
\label{sec:evo-DM-related}
Historically, evolution provenance was mainly linked to scientific workflows where it was used to document the (development) process. 
Thereby, it automatically tracks the changes made between two versions that can be inputs, parameters, or functions invoked. 
Callahan et al proposed the first approach~\cite{Callahan06} that captures this kind of evolution provenance.
The rationale behind that was to accelerate the decision process about the correct module semantics since multiple workflow executions can be compared visually using the proposed tool called VisTrails.
In the same context, Ellkvist et~al.~\cite{Ellkvist:spri08} propose a solution that facilitates collaboration between developers. 
It tracks changes made by all users in realtime during a collaborative design of the workflow and renders them in the form of branches. 
This particular type of provenance obtained in the two aforementioned approaches~\cite{Callahan06,Ellkvist:spri08} is known as ``workflow provenance'' or ``provenance of the development process". 

We use the term ``evolution provenance'' proposed  in~\cite{Herschel2017survey} as it was shown that this particular type of provenance also encompasses other applications besides scientific workflows.
Indeed, unlike aforementioned work~\cite{Callahan06} and~\cite{Ellkvist:spri08} where tracking the evolution provenance in scientific workflow engines is only limited to changes of the workflow specification, we capture in this thesis the evolution provenance for visual data exploration to describe an actual run of a visual exploration session, divided into individual exploration steps. 




In the same context, we find several existing work (e.g.,~\cite{Milo:2016,2016_eurovis_clue}) that collect evolution provenance for visual data exploration processes.
For instance, REACT~\cite{Milo:2016} computes evolution provenance and uses it to compute collaborative-filtering query recommendations.
CLUE~\cite{2016_eurovis_clue}, a history based visual exploration system captures the evolution provenance of a visual data exploration process and offers users the capability to assemble states of interest (i.e., interesting explorations tasks) into a story that can be used to explain or to remember exploration tasks made to reach important observations.



Our review of evolution provenance data models adopted in existing visual data exploration work, (e.g.,~\cite{Milo:2016,2016_eurovis_clue}) shows that these work focus either on collecting visualization-related properties or on collecting query-related properties.
More specifically, REACT~\cite{Milo:2016} proposes an evolution provenance data model that stores information about inspected queries and navigations types (e.g., data retrieval, cube operations, and data mining) whereas it neglects completely the storage of visualizations information rendered in the course of the exploration.
Opposed to that, CLUE~\cite{2016_eurovis_clue} offers an evolution provenance data model that focuses mainly on the collection of visualization-related properties. Accordingly, the evolution provenance model proposed in CLUE~\cite{2016_eurovis_clue} stores users' interactions and  visual encodings of rendered visualizations. %properties including visualization techniques and visualization resources.
Note that CLUE~\cite{2016_eurovis_clue}'s evolution provenance data model has a very limited support to query-related properties. Hence, this model stores inspected data sets. This is not convenient when inspecting large datasets and may incur a costly process of evolution provenance collection and storage.


In what follows, we propose a novel evolution provenance model that ensures the collection of 
visualization-related properties as well as query-related properties.



%%OLD VERSION
%\label{sec:evo-DM-related}
%Historically, evolution provenance was mainly linked to scientific workflows where it was used to document the (development) process. 
%Thereby, it automatically tracks the changes made between two versions that can be inputs, parameters, or functions invoked. 
%Callahan et al proposed the first approach~\cite{Callahan06} that captures this kind of evolution provenance.
%The rational behind that was to accelerate the decision process about the correct module semantics since multiple workflow executions can be compared visually using the proposed tool called VisTrails.
%In the same context, Ellkvist et~al.~\cite{Ellkvist:spri08} propose a solution that facilitates collaboration between developers. 
%It tracks changes made by all users in realtime during a collaborative design of the workflow and renders them in the form of branches. 
%{\color{Fuchsia}This particular type of provenance obtained in the two aforementioned approaches~\cite{Callahan06,Ellkvist:spri08} is known as ``workflow provenance'' or ``provenance of the development process". }
%
%We use the term ``evolution provenance'' proposed  in~\cite{Herschel2017survey} as it was shown that this particular type of provenance also encompasses other applications besides scientific workflows.
%Indeed, unlike aforementioned work~\cite{Callahan06} and~\cite{Ellkvist:spri08} where tracking the evolution provenance in scientific workflow engines is only limited to changes of the workflow specification, we capture in this thesis the evolution provenance for visual data exploration to describe an actual run of a visual exploration session, divided into individual exploration steps. 
%
%
%
%{\color{Fuchsia}
%In the same context, we find several existing work (e.g.,~\cite{Milo:2016,2016_eurovis_clue}) that collect evolution provenance for visual data exploration processes.
%For instance, REACT~\cite{Milo:2016} computes evolution provenance and uses it to compute collaborative-filtering query recommendations.
%CLUE~\cite{2016_eurovis_clue}, a history based visual exploration system captures the evolution provenance of a visual data exploration process and offers users the capability to assemble states of interest (i.e., interesting explorations tasks) into a story that can be used for presentation and recall applications already discussed in Section~\ref{sec:prov-purposes}.
%CLUE~\cite{2016_eurovis_clue} fosters also reproducibility of an analysis process by letting the user resume exploration from a specific state stored in the evolution provenance.
%
%In the present manuscript, we adopt a similar approach that leverages the evolution provenance to foster reproducibility of results, to provide stories about successful explorations, as well as to assist users with collaborative-filtering query recommendations.
%
%
%
%Our review of evolution provenance data models adopted in existing visual data exploration work, (e.g.,~\cite{Milo:2016,2016_eurovis_clue}) shows that these work focus either on collecting visualization-related properties or on collecting query-related properties.
%More specifically, REACT~\cite{Milo:2016} proposes an evolution provenance data model that stores information about inspected queries and navigations types (e.g., data retrieval, cube operations and data mining) whereas it neglects completely the storage of visualizations information rendered in the course of the exploration.
%Opposed to that, CLUE~\cite{2016_eurovis_clue} offers an evolution provenance data model that focuses mainly on the collection of visualization-related properties. Accordingly, the evolution provenance model proposed in CLUE~\cite{2016_eurovis_clue} stores users' interactions and  visual properties including visualization techniques and visualization resources.
%Note that CLUE~\cite{2016_eurovis_clue}'s evolution provenance data model has a very limited support to query-related properties. Hence, this model stores inspected data sets. This is not convenient when inspecting large datasets and may incur a costly process of evolution provenance collection and storage.
%
%
%In what follows, we propose our proper evolution provenance model that ensures the collection of 
%visualization-related properties as well as query-related properties.



	

\section{Evolution provenance data model representation}
\label{sec:evo-core}

We have introduced in Chapter~\ref{chap:overview} \framework{}, our visual data exploration framework. Overall, it is clear from this description that our framework \framework{} offers users the possibility to perform a set of consecutive exploration steps where they explore visually at each step a different region of the data. More precisely, we define an exploration step as follows. 

  \subsection{Exploration step}     
  \label{subsec:explorStep}
\begin{definition}[Exploration step] 
\label{def:expo-step}An exploration step over data in $D$ is defined as $X^D=\{Q,V\}$ where $Q$ is the exploration query whose result $Q(D)$ over dataset $D$ is rendered with an interactive visualization described by $V$. 
\end{definition}


 
 Inspired by~\cite{WuPMZR17}, we record meta-data about \emph{visual exploration resources} used to render visualizations in a relational database. The design of this schema draws on a set of visualization specifications such as Vega~\cite{vega17} and D3~\cite{2011-d3}, which define a visualization using four fundamental components. (1)~The \emph{scale} component maps data values to visual values, e.g., specification of position, color or shape encodings for such data value. (2)~The \emph{axis} component provides a reference for reading the visual/data mapping defined by the \emph{scale} component. (3)~Graphical \emph{marks} describe the graphical forms adopted to visually represent data, e.g., circles, rectangles, or points. (4)~Finally, a given graphical form is further specified using a set of encoding \emph{channels} that describe the visual mapping of each attribute.  In addition to these main components, we further store information about data to render, or the width and the height of a visualization. 
 

\sloppy

 This gives rise to the following relational schema depicted in Figure~\ref{fig:VisSchema}.
 \begin{figure}[b]
    {\fontfamily{cmt}\selectfont
    $Visualization(\underline{idVisualization}, width, length,idQuery), \newline
     Mark(\underline{idMark}, markType, idVis \rightarrow Visualization), \newline
     AxisUsage(\underline{idAxisUsage}, typeUsage, idAxis \rightarrow Axis, \newline$ \hspace*{1.6cm} $idVis \rightarrow Visualization) \newline
     Axis(\underline{idAxis}, tiltle, tick, type, idScale \rightarrow Scale) \newline
     Scale(\underline{idScale}, typeScale, nameScale, fieldDom, fieldRange,  \newline$ \hspace*{0.9cm} $Literal, idDataset \rightarrow Dataset) \newline
       Dataset(\underline{idDataset}, value, name, transformation, Literal, source) \newline
      Channel(\underline{idChannel}, typeChannel, typeChannel, idScale \rightarrow Scale) \newline
      ChannelUsage(\underline{idChannelUsage}, typeUsage, idChannel \rightarrow Channel, \newline$ \hspace*{2.2cm} $ idMark \rightarrow Mark) %\newline
     $
}
\caption{$Sch_{VIS}$ : the relational schema modelling visualizations}
\label{fig:VisSchema}
\end{figure}    
   

In this relational schema, we present first the name of each relation. Furthermore, we list the set of attributes present in each relation.
Note that underlined attributes refer to the primary keys of each proposed relation while arrows refer  to the foreign keys.

In what follows, we use $Sch_{VIS}$ to refer to this relational schema (depicted in Figure~\ref{fig:VisSchema}).
This relational schema $Sch_{VIS}$ stores separately information about \emph{Visualization}, \emph{Axis}, \emph{Scale}, \emph{Mark}, and \emph{Channel}. Given that an axis may be used by more than one visualization in the course of the visual data exploration process, $Sch_{VIS}$ encompasses a table \emph{AxisUsage} that stores information about relationships between visualization and used axis. Similarly, $Sch_{VIS}$ contains a table \emph{ChannelUsage} that stores relationships information between encoding channels and graphical marks given that an encoding channel may be used by many graphical marks.

Based on this design of visualization resources, a visualization $V$ generated in an exploration step is defined as a query over $Sch_{VIS}$.
%, the schema modelling visual exploration resources. 
The query that specifies a visualization $V$ is defined as follows.

\begin{definition}[Visual exploration resource] 
\label{def:vis-resource}
A visualization $V$ for an exploration step $X^D=\{Q,V\}$  is defined as the following query over %the schema depicted in Figure~\ref{fig:VisSchema}.
the relational schema $Sch_{VIS}$.
\begin{eqnarray*}
V &= &  \Pi_{Vis.*, Scale.*, Dataset.*, Mark.*, Channel.*, Axis.*}(\\
& & \sigma_{query = Q}(Visualization) \Join Mark \Join ChannelUsage\\
& & Channel \Join Scale \Join Dataset  \Join AxisUsage \Join Axis)
\end{eqnarray*}
\end{definition} 



\subsection{Exploration path} 
\label{sec:expo-path}
As described in Section~\ref{sec:frame-navi}, the user can navigate using our framework \framework{}  from an exploration step to another to acquire more knowledge.
These navigations result in exploration paths defined as follows.



 \begin{definition}[Exploration path] \label{explo:path} 
We denote the exploration path that leads to the currently investigated exploration step $X_{curr}^D=\{Q_{curr},V_{curr}\}$ as $P_X = [X_0^D, X_1^D, \ldots, X_k^D, X_{curr}^D]$ where $P_X$ encompasses all $X_i^D$  $\forall i \in [0,n]$ explored by the user, and $P_X$ is the path from the seed exploration step 
  $X_0^D$ until reaching $X_{curr}^D$.
\end{definition}


	\subsection{Evolution provenance graph}
	\label{subsec:evo}

The automata depicted in Figure~\ref{fig:automata} shows that the user can enjoy using our framework \framework{} a flexible navigation model to explore the data. This results in many exploration paths (see Definition~\ref{explo:path}). These paths can be gathered  to get the full story about the user's manipulations over an exploration session.
% To do that, we propose a new structure to store the flow of information generated within the automata in understandable way.
%As it describes the evolution of the visual data exploration process,  we refer to this new structure as \emph{the evolution provenance} in what follows.
The result is an \emph{evolution provenance} graph. As its name indicates, the evolution provenance graph tracks  the evolution of the visual data exploration process in the course of a user's exploration session.
%The evolution provenance is a graph that tracks 
Hence, it records a user's exploration session including, for each exploration step, meta-data about the explored data, the query issued to reach this particular region of data, and the visualization resources used to render the explored data. Overall, we define the evolution provenance as follows.\\

%\begin{definition}[Evolution provenance graph]
%\label{def:session}
%An evolution provenance graph describes an exploration session over a {\color{Fuchsia}dataset } $D$.
%It  is a labeled directed acyclic graph (DAG) $\sessionGraph{}_{,D}(\sessionV{}, \sessionE{})$ where $\sessionV{}$ is a set of nodes and $\sessionE{}$ a set of labeled edges. 
%Each node $n \in \sessionV$ corresponds to an exploration step $X$.
%An edge $e = (n, n', \sessionL{})$ represents the transition from one exploration step $X_D = \{Q, V\}$ to the next exploration step $X'_D = \{Q', V'\}$ whose query $Q'$ is a recommendation directly derived from $Q$ over the same $D$. $\sessionL{}$ is a 3-tuple $\langle \labelOp{}, a, s \rangle$ where $\labelOp{}$ is an identifier of the query type of $Q'$ wrt $Q$, $a$ is the relevant attribute used to construct $Q'$ based on $Q$, and $s$ an impact score computed as $s(e) = 1+\sum_{e_c = (n', n_i, \sessionL') \in E} s(e_c)$.
%\end{definition}	 
\begin{definition}[Evolution provenance graph]
\label{def:session}
An evolution provenance graph describes an exploration session over a dataset  $D$.
It is a labeled directed acyclic graph (DAG) $\sessionGraph{}_{,D}(\sessionV{}, \sessionE{})$ where $\sessionV{}$ is a set of nodes and $\sessionE{}$ a set of labeled edges. 
Each node $n \in \sessionV$ corresponds to an exploration step $X$.
An edge $e = (n, n', \sessionL{})$ represents the transition (see navigations in Section~\ref{sec:frame-navi}) from one exploration step $X_D = \{Q, V\}$ to the next exploration step $X'_D = \{Q', V'\}$ whose query $Q'$ is a recommendation directly derived from $Q$ over the same $D$. $\sessionL{}$ is a 2-tuple $\langle const, s \rangle$ where $const$ is metadata that describes the derivation process of $Q'$ from $Q$, and $s$ is an impact score computed as $s(e) = 1+\sum_{e_c = (n', n_i, \sessionL') \in E} s(e_c)$.
\end{definition}	 


%Note that when $D$ is clear from the context, we omit in what follows the subscript $D$ from $X_D$ and $\sessionGraph{}_{,D}$.
Note that when $D$ is clear from the context, we omit the subscript $D$ from $X_D$ and $\sessionGraph{}_{,D}$.

Based on this definition, the evolution provenance graph represents the full exploration session made by a user. 
The utility of each edge (navigation) is measured using a score $s(e)$.
This score reflects the impact of the transition from an exploration step to another on the rest of the exploration session. 
Accordingly, the score $s(e)$ is computed as the number of subsequent exploration steps stemming from this navigation $e$ augmented with one.


%{\color{Fuchsia}
Note that the DAG property is an important aspect of the evolution provenance graph. Indeed, using the direction of edges, we can understand the impact of each performed exploration step on the acquirement of knowledge in the rest of the exploration session. Accordingly, we leverage this information as we will show later to recommend exploration steps having high impacts, to users exploring later the same dataset.
%}
 The direction property of the DAG is also useful for the storytelling process~\cite{2016_eurovis_clue} where an analyst needs to explain to an audience how such insight was discovered.
Finally, we point out that the acyclicity property ensured by the DAG, is also an important property in the context of visual data exploration as it maintains the rationale of recommendation by prohibiting suggesting exploration steps already investigated by the user.

		
	

		
\begin{example}[Sample evolution provenance graph]
\label{ex:expo-session}
Figure~\ref{fig:session} shows an example of evolution provenance graph. It describes a user's exploration session made using \prototype{}.%, the instance of our framework meant to explore data warehouses.
Nodes in this graph represent exploration steps. Edges in this graph contain two labels: (i)~the first refers to the type of recommended query and the attribute-value pair used do construct the recommendation (cf.~Example~\ref{ex:sampleMatrix}), and (ii)~the second label is the score referring to the number of subsequent explorations reached when traversing this edge.
The user initially queries the set of delayed flights as described in Examples~\ref{ex:query} and~\ref{ex:VIS}. Later, the user double clicks on state California (see Example~\ref{ex:interaction}) as it has the highest number of delayed flights. This triggers the computation of recommendation visualized in Example~\ref{ex:sampleMatrix}. 

At this stage, the user selects first the \emph{Extension} recommendation with airport of destination query associated to the  $\{(airline\_Id, \{WN,UA,OO,AS\})$ pair. This leads to the generation of node $X_1$ and an edge $e$ from $X_0$ to $X_1$. These information are stored in the evolution provenance graph.   
Now let us assume that the user goes one step back to explore other recommendations visualized in Example~\ref{ex:sampleMatrix}.  In this case, the edge $e$ from $X_0$ to $X_1$ has an impact score equal to 1 as the exploration step $X_1$ has no subsequent steps. 
Now, the user inspects two recommendations related to the exploration step $X_0$. This leads to the generation of two new navigations (edge from $X_0$ to $X_2$ and edge from $X_0$ to $X_3$) in the evolution provenance graph. 
Subsequently, the user continues exploring information related to the result inspected in the step $X_3$. Accordingly, the user navigates to the exploration step $X_4$. 
Based on user's interactions, the score labeling the edge from $X_0$ to $X_3$  is set to 2, to reflect the number of exploration steps, pursued upon reaching $X_3$, indicating thereby how many potentially interesting steps can be reached from this exploration step.
\end{example}

\begin{figure}[t]
\centering
\includegraphics[scale=0.52]{figures/evoDM/session.pdf}
\caption{Sample evolution provenance graph}
\label{fig:session}
\end{figure}


Now, we discuss the mapping process performed to model the flow of information incurred from the automata depicted in Figure~\ref{fig:automata} following our evolution provenance data model.
This relies mainly on inference rules depicted in Figure~\ref{fig:Evoinf}.




Essentially, our evolution provenance graph maintains a pointer that we call \emph{context}. This latter presents the current entry to expand the provenance graph. Accordingly, any subsequent exploration step performed later, will be considered as a direct descendant of the \emph{context}.
Our evolution provenance graph is initialized by a first exploration step as specified in the \emph{Initialization} inference rule (shown in Figure~\ref{fig:Evoinf}). This first exploration step is introduced in the evolution provenance graph as a node. It is then designated as a \emph{context}. 
This latter is updated in the course of the exploration session via inference rules \emph{context update1} and \emph{context update2} when the user performs a navigation of type \emph{interact} or of type \emph{recover} (cf.~Figure~\ref{fig:automata}).
Finally, the \emph{getNextRec} navigation (cf.~Figure~\ref{fig:automata}) is responsible for the generation of new edges and nodes that will be added to the evolution provenance graph.
Hence for this particular type of navigation, the user inspects the set of recommendations triples ($L_{T_i}$) output by our framework and selects a particular triple $T_i$ (corresponding to a new recommended exploration step $ES_{new}$) to study next. This leads to the inference of a new node added to the evolution provenance graph (see inference rule \emph{node inference}).
%{\color{Fuchsia}
Similarly, a new edge $(context \rightarrow ES_{new})$ between the \emph{context} and the new node corresponding to $ES_{new}$ is added to the evolution provenance graph.
This is done using the inference rule \emph{edge inference} described in Figure~\ref{fig:Evoinf}.
%}



\begin{figure}[t]
 \centering
\scalemath{0.8}{
 \begin{tabular}{lll}

%$E' = \left(E' \setminus \{(p, n, l)\} \right) \cup \{ (p, N_{merged}, updateLabel()) \} $
 \multicolumn{1}{c}{
 \staterule{(\textsc{Initialization})}
  {evolution graph = \emptyset}
 %{  node (ES_{i}), context= ES_{i}   }
{  V \leftarrow V \cup \{ES_{i}\}, context  \leftarrow  ES_{i} }
}
   &
      \multicolumn{1}{c}{\staterule{(\textsc{context inference})}
 {Interact(ES_i)=L_{T_i} }
%{  context= ES_{i}   }
{ context  \leftarrow  ES_{i} }
 }
  \\ \\
   \multicolumn{1}{c}{\staterule{(\textsc{context update1 })}
 {recover(ES_i)=ES_{i-k} }
%{  context= ES_{i-k}   }
{ context  \leftarrow  ES_{i-k}  }
 }

&
   \multicolumn{1}{c}{\staterule{(\textsc{context update2})}
% {recover(M_i)=ES_{i-k} }
  {recover(L_{T_i})=ES_{i-k} }
%{  context= ES_{i-k}   }
{ context  \leftarrow  ES_{i-k}  }
 }
  \\ \\
      \multicolumn{1}{c}{\staterule{(\textsc{Node inference})}
% {getNextRec(M_i)=ES_{new} }
  {getNextRec(L_{T_i},T_j)=ES_{new} }
% {node (ES_{new})} 
{  V \leftarrow V \cup \{ES_{new}\}}
 }
&
      \multicolumn{1}{c}{\staterule{(\textsc{Edge inference})}
        {getNextRec(L_{T_i},T_j)=ES_{new} }
% {getNextRec(M_i)=ES_{new} }
 %{edge(context,ES_{new})} 
 {  E \leftarrow E' \cup \{(context \rightarrow ES_{new})\}}
 }
 \end{tabular}
}

 \caption{\label{fig:Evoinf}Evolution provenance graph inference rules.}
 \end{figure}



Finally, we point out that the evolution provenance graph is stored in a relation database alongside, the visualization resources discussed previously in Section~\ref{subsec:explorStep}.


 



	\subsection{Multi-user graph}  
%{\color{Fuchsia}
The evolution provenance graph above describes the visual data exploration process of a single user. As we shall see when discussing our recommendation algorithms, a graph summarizing multiple exploration sessions made by multiple users can %provide further valuable information. 
support recommendations based on collaborative-filtering.
We therefore define such summarized graph, called multi-user exploration graph next as follows.
%}

%\hou{check that we are using $\usersV{}$ everywhere}
\begin{definition}[Multi-user exploration graph]
\label{def:MUg}
%A multi-user exploration graph $\usersGraph{}(\usersV{}, \usersE{})$ is the result of merging many evolution provenance graphs $\{\sessionGraph{}_1, \ldots, \sessionGraph{}_n\}$. 
%Formally, a multi-user exploration graph $\usersGraph{}(\usersV{}, \usersE{}) =  \bigcup_{i=1}^n  \sessionGraph{}_i$ where each $\sessionGraph{}_i=(R_{XSi} ,E_{XSi} )$ is an evolution provenance graph.
%$G_{MU}$ has a surjective map $M$: $ \bigcup_{i=1}^n R_{XSi} \longrightarrow \usersV{}$ such that:\begin{itemize}
%\item  $\forall e=(n_1,n_2,L) \in E_{XSi},  \exists$ $e^*=(n^*_1,n^*_2,L^*) \in E_{MU}$ such that the metadata of the two edges are similar, i.e. $e.L[const]=e^*.L^*[const]$ and we have two matchings: $M(n_1)=n^*_1$ and $M(n_2)=n^*_2$.
%\item  $\forall e^*=(n^*_1,n^*_2,L^*) \in E_{MU}$, there exists at least one evolution provenance session graph $\sessionGraph{}_i$ with $n_1,n_2 \in N_{XSi}$ such that $M(n_1)=n^*_1$, $M(n_2)=n^*_2$, $e.L[const]=e^*.L^*[const]$ and $(n_1,n_2,L) \in E_{XSi}$.
%\item  $\forall n_1 \in R_{XSi}$ and $n_2 \in N_{XSj}$, $M(n_1)=M(n_2)$ implies $i \neq j$.
A multi-user exploration graph $\usersGraph{}(\usersV{}, \usersE{})$ is the result of merging many evolution provenance graphs $\{\sessionGraph{}_1, \ldots, \sessionGraph{}_n\}$. 
Formally, a multi-user exploration graph $\usersGraph{}(\usersV{}, \usersE{}) =  \bigcup_{i=1}^n  \sessionGraph{}_i$ where each $\sessionGraph{}_i=(N_{XSi} ,E_{XSi} )$ is an evolution provenance graph.
$G_{MU}$ has a surjective map $M$: $\bigcup_{i=1}^n \sessionGraph{}_i \longrightarrow \usersGraph{}$ such that:
\begin{itemize}
\item  $\forall n \in N_{XSi},  \exists$ $n^* \in \usersV{}$ such that $M(n)=n^*$.
\item  $\forall e=(n_1,n_2,L) \in E_{XSi},  \exists$ $e^*=(n^*_1,n^*_2,L^*) \in E_{MU}$ such that the metadata of the two edges are similar, i.e. $e.L[const]=e^*.L^*[const]$ and we have two matchings: $M(n_1)=n^*_1$ and $M(n_2)=n^*_2$. 
\item  $\forall n_1 \in N_{XSi}$ and $n_2 \in N_{XSj}$, $M(n_1)=M(n_2)$ implies $i \neq j$.
\end{itemize}

\end{definition}  
The first two properties of the surjective map $M$ ensures that each edge (similarly for node) in an input evolution provenance graphs $\sessionGraph{}_i$ has necessary a corresponding edge (similarly for node) in the multi-user exploration graph $\usersGraph{}$. 
Finally, the third property of the surjective map $M$ ensures that no two nodes belonging to the same input evolution provenance graph map to the same target node in $\usersGraph{}$.


In other words, the multi-user exploration graph of a collection of evolution provenance graphs is a single labeled directed acyclic graph 
where each node (edge) represents many nodes (edges), each belonging to a different input evolution provenance graph $\sessionGraph{}_i$. $|M^{-1}(n)|$ denotes the set of nodes in $\bigcup_{i=1}^n \sessionGraph{}_i$ that are mapped to a node $n \in \usersV{}$.
The connectivities and the directions defined in each input $\sessionGraph{}_i$ are maintained in the $\usersGraph{}$.
 The multi-user exploration graph follows the definition of evolution provenance graphs (see Definition~\ref{def:session}). We merely adjust the definition of edge scores. 
For an edge $e = (n', n, \sessionL{})$, we compute its score as $ s(e) = 1+|M^{-1}(n)|+\sum_{e_c = (n, n_i, \sessionL) \in E} s(e_c)$. Intuitively, $s(e)$ reflects the frequency of $n$ (the number of exploration steps in the individual graphs that are mapped to $n$) as well as the number of exploration steps reached by this navigation $e$.




\begin{example}[Example of multi-user graph]
\label{ex:multi-userG}
Consider another user exploring the same region of data studied in Example~\ref{ex:expo-session} using \prototype{} the instance of our framework \framework{}. The evolution provenance tracking the exploration session of this second user is depicted in Figure~\ref{fig:session2}.

%Towards constructing a multi-user graph, we merge the evolution provenance graph shown in Figure~\ref{fig:session2} with the evolution provenance graph discussed in Example~\ref{ex:expo-session}. Assume in this scenario that our merge approach fuses the following exploration step pairs \{($X_0,X_0'$), ($X_{1},X'_{1}$), ($X_{4},X'_{2}$), and ($X_{3},X'_{4}$)\} into nodes $X_{00}$, $X_{11}$, $X_{42}$, and $X_{34}$ respectively.
%The resulting multi-user graph is depicted in Figure~\ref{fig:gmu}.
%Merged exploration steps $ES_{merged}$ are highlighted in red.
%{\color{Fuchsia}Accordingly, edges $e_{merged}$ pointing to $ES_{merged}$ are also updated (e.g., edge pointing to the node $X_1$). More precisely, we augment impact scores of $e_{merged}$ to reflect the frequency of each item present in $ES_{merged}$}. Consequently, scores of edges pointing to merged exploration steps  $X_{11}$ , $X_{42}$ and $X_{34}$ are increased. 
Towards constructing a multi-user graph, we merge the evolution provenance graph shown in Figure~\ref{fig:session2} with the evolution provenance graph discussed in Example~\ref{ex:expo-session}. Assume in this scenario that our merge approach fuses the following exploration step pairs \{($X_0,X_0'$), ($X_{1},X'_{1}$), and ($X_{2},X'_{2}$)\} into nodes $X_{00}$, $X_{11}$, and $X_{22}$ respectively.
The resulting multi-user graph is depicted in Figure~\ref{fig:gmu}.
Merged exploration steps $ES_{merged}$ are highlighted in red.
Accordingly, edges $e_{merged}$ pointing to $ES_{merged}$ are also updated (e.g., edge pointing to the node $X_1$). More precisely, we augment impact scores of $e_{merged}$ to reflect the frequency of each item present in $ES_{merged}$. Consequently, scores of edges pointing to merged exploration steps   $X_{11}$ and $X_{22}$ are increased. 
\end{example}
\begin{figure}[t]
\centering
\includegraphics[scale=0.45]{figures/evoDM/session2.pdf}
\caption{Another example of an exploration session graph}
\label{fig:session2}
\end{figure}


\begin{figure}[t]
\centering
\resizebox {0.8\textwidth} {!} 
	{
  \begin{tikzpicture}[-latex, auto, node distance = 4 cm and 5cm, on grid, semithick, state/.style ={circle, top color = white, bottom color = processblue!20, draw, processblue, text = blue, minimum width = 0.7 cm}, fused_state/.style = {circle, top color = white, bottom color = antiquebrass!40, draw, antiquebrass, text = black, minimum width = 0.7 cm}, edge_style/.style = {draw = processblue!60, dashed}, alt_state/.style = {circle,  top color = white, bottom color = processblue!60, draw, Emerald, text = black, minimum width = 0.7 cm},alt_state2/.style = {circle,  top color = white, bottom color = Emerald!60, draw, Emerald, text = black, minimum width = 0.7 cm}]


						\node[fused_state] (X0) at (-2, 9.2) {$X_{00}$};
						\node[fused_state] (X1) at (-2, 2.5) {$X_{11}$};
						\node[fused_state] (X2) at (3.8, 4.2) {$X_{22}$};
						\node[alt_state2] (X4) at (9.5, 8) {$X_{4}$};	
						\node[alt_state2] (X3) at (4.2, 8.2) {$X_{3}$};
						\node[alt_state] (X'3) at (10, 5)  {$X'_{3}$};
						
						\node[alt_state] (a) at (12, 5)  {$$};
						\node[alt_state2] (a) at (12, 6)  {$$};
						\node[fused_state] (a) at (12, 7)  {$$};
						\node[text width=2.5cm] at (14, 5) {   nodes from $G_1$};
						\node[text width=2.5cm ] at (14, 6) {  nodes from $G_2$};
						\node[text width=2.5cm] at (14, 7) {  merged nodes};

					\path [every node/.style={sloped,anchor=south,auto=false}] (X0) edge node[above,sloped] {$Extension-airport(airlineId),2>$} (X1);
						\path [every node/.style={sloped,anchor=south,auto=false}] (X0) edge node[above,sloped] {$<Extension-airport(airlineId),3>$} (X2);
						\path [every node/.style={sloped,anchor=south,auto=false}] (X0) edge node[above,sloped] {$<Zoom-In Slice(airlineId),2>$} (X3);
			
						\path [every node/.style={sloped,anchor=south,auto=false}] (X3) edge node[above,sloped] {$<Zoom-In Slice(year),1>$} (X4);

					\path [every node/.style={sloped,anchor=south,auto=false}] (X2) edge node[above,sloped] {$<Zoom-In Slice(distance),1>$} (X'3);

						
			
					\end{tikzpicture}	
					}
%\caption{Example of multi-user graph (with nodes in green color belong to graph $G_1$ depicted in Figure~\ref{fig:session}, nodes in blue belong to graph $G_2$ depicted in Figure~\ref{fig:session2}, and nodes in red color are result of the merge process)}
\caption{Example of multi-user graph}
\label{fig:gmu}
\end{figure}



%\section{Evolution provenance dissemination in visual computing}
%\label{sec:evo-instances}
Our evolution provenance data model is adopted in the two instances of our framework {\color{Fuchsia}\framework{}}. 
The first instance of our framework is \prototype{}~\cite{BHBK18:Demo} that offers a visual exploration of data warehouses.
The second instance of our framework is the volume-based large dynamic graph analysis solution~\cite{Bruder2019} that offers an interactive visual exploration of large dynamic graphs.


Note that, alongside the description of the evolution data model in Section~\ref{sec:evo-core}, we provided several examples (e.g., Examples~\ref{ex:query-path},~\ref{ex:expo-session}, and~\ref{ex:multi-userG}) showing the feasibility of evolution provenance computation in \prototype{}.


Therefore, we dedicate the rest of this section to discuss the incorporation of our evolution provenance data model into the second instance of our framework that allows for interactive analysis of dynamic graphs. 



\subsection{Overview of large dynamic graph exploration}\label{A02:overview}
%Bruder et al.~\cite{bruder:18} propose recently a visual analytics approach that allows for interactive analysis of dynamic graphs containing several thousand time steps and hundreds of nodes.
%	
%
%Inspired by space-time cube approaches e.g.,~\cite{bach:2017} and~\cite{Bach_CHI:14}, the proposed approach adopts a volumetric representation of the dynamic graph based on its adjacency matrices.
%These concepts have in common that they stack representations of individual time steps to gain a three dimensional data structure.
%As illustrated in Figure~\ref{fig:illu}, adjacency matrices are 2D structures that are stacked to incorporate the temporal evolution of the graph.
%In the resulting three dimensional cuboid structure, the $x$- and $y$-axes represent nodes, while entries in the plane defined by those two axes represent edges (including their weights). 
%The $z$-axis represents time.
%
%\begin{figure}
%	\centering
%	\includegraphics[width=1.0\linewidth]{figures/evoDM/stacking-crop}
%	\caption{Sample of explored adjacency matrices~\cite{Bruder2019}}
%	\label{fig:illu}
%\end{figure}
%
%
%
%This cuboid is a concise and static representation of the whole dynamic graph that preserves the mental map~\cite{Misue:95,Archambault:11}. 
%To allow for fluent user interactions (e.g., rotating and zooming into the cuboid or filtering values), GPU-accelerated volume rendering methods are employed to generate the graph data visualizations. 
%Using those techniques, the proposed approach~\cite{bruder:18} is capable to render several thousand time steps and hundreds of nodes at the same time without loosing interactivity.
%
%
%This approach offers a set of analytic methods, used to perform typical analysis tasks such as detecting temporal patterns, e.g., clusters forming or repeating occurrences of similar node-link structures.
%Essentially, we distinguish three classes of analytic methods.
%
%\begin{description}
%	\item[Data Views.] 
%	Large dynamic graphs typically exhibit a lot of information covering different aspects of interest. 
%	A single visualization is often not capable to convey them all.
%	Therefore, it is important to offer distinct perspectives on the data using multiple views with suitable visualizations. 
%	The volume view is the primary visualization adopted by this approach.  Indeed, it provides a direct visualization of the full graph volume. This latter could be split into sub-volumes along the time axis. This is possible using the volume partitioning perspective.
%	Other views include for instance the timeline plot used to show different graph metrics on a 2D plot over time. 
%	This approach offers also detailed visualization such as Slice view that provides information of individual, selected time steps with 2D slice  views. 
%	\item[Aggregation and Filtering.] 
%	Showing all edges of a large dynamic graph, especially a dense one, quickly leads to visual clutter, overload, and occlusion using our volumetric approach. 
%	This makes filtering and aggregation of adjacent edges an essential part of the analysis process to reduce the visualization to relevant information.
%	Examples of aggregation functions include for instance minimum, maximum, average weight and density of edges. These methods could be applied either on in the spacial domain i.e. aggregating neighboring edges or in the time domain.
%	Filtering methods consists of omitting edges not satisfying such predicate e.g. filter out edges with low density.
%	\item[Comparison.] 
%	Comparing different sections within a temporal graph or even several distinct dynamic graphs becomes challenging using the methods discussed above. 
%	The support of a dedicated visualization for this specific task is therefore important for the analysis process.
%	This is possible using this approach. Indeed, an analyst can select starting points and a range of time steps to compare them against one another.
%	In this case, a matrix view is generated to render commonality and irregularities resulting from comparing the different time sequences.
%\end{description}

The volume-based large dynamic graph analysis solution is an instance of our visual data exploration framework {\color{Fuchsia}\framework{}}.
It is based on a volume-based approach proposed in~\cite{Bruder2019} where we propose methods to allow for interactive analysis of dynamic graphs containing several thousand time steps and hundreds of nodes.
	

%Inspired by space-time cube approaches e.g.,~\cite{bach:2017} and~\cite{Bach_CHI:14}, authors in~\cite{Bruder2019} adopt a volumetric representation of the dynamic graph based on its adjacency matrices.
%These concepts have in common that they stack representations of individual time steps to gain a three dimensional data structure.
%As illustrated in Figure~\ref{fig:illu}, adjacency matrices are 2D structures that are stacked to incorporate the temporal evolution of the graph.
%In the resulting three dimensional cuboid structure, the $x$- and $y$-axes represent nodes, while entries in the plane defined by those two axes represent edges (including their weights). 
%The $z$-axis represents time.
Inspired by space-time cube approaches (e.g.,~\cite{bach:2017} and~\cite{Bach_CHI:14}), the instance of our framework adopts a volumetric representation of the dynamic graph.
As illustrated in Figure~\ref{fig:illu}, individual time steps of the dynamic graph are initially represented as adjacency matrices.
Those latter are stacked into space-time cubes.
This results in three dimensional volume structure where the $x$- and $y$-axes represent nodes, while entries in the plane defined by those two axes represent edges (including their weights). 
The $z$-axis represents time.

\begin{figure}
	\centering
	\includegraphics[scale=0.3]{figures/evoDM/stacking-crop}
	\caption{Sample of explored adjacency matrices~\cite{Bruder2019}}
	\label{fig:illu}
\end{figure}


To allow for fluent user interactions, the instance of our framework~\cite{Bruder2019} offers a set of analytic methods, used to explore large dynamic graphs.
Essentially, we distinguish three classes of analytic methods {\color{Fuchsia}(equivalent to exploration queries in our framework)}.

\begin{description}
	\item[Data Views.] 
A single visualization is often not capable to convey all information of the exhibited large dynamic graphs.
Therefore, several data views are proposed to visualize large dynamic graphs.
This includes for instance the volume view, the timeline plot and slice views that are detailed in~\cite{Bruder2019}.
	\item[Aggregation and Filtering.] 
Filtering and aggregation of adjacent edges an essential part of the analysis process to reduce the visualization of large dynamic graphs to relevant information.
Examples of aggregation functions include for instance minimum, maximum, average weight and density of edges. These methods could be applied either on in the spacial domain i.e. aggregating neighboring edges or in the time domain.
Filtering methods consists of omitting edges not satisfying such predicate, e.g., filter out edges with low density.
\item[Comparison.] 
Comparing different sections within a temporal graph or even several distinct dynamic graphs is supported via the comparison function. 
In this case, a matrix view is generated to render commonality and irregularities resulting from comparing the different time sequences or distinct dynamic graphs.
\end{description}

		
\subsection{Evolution provenance model}\label{A02:evoDM}
Using our volume-based large dynamic graph analysis solution~\cite{Bruder2019},
users are engaged in a \emph{visual analysis session} where they perform various \emph{visual analysis steps} iteratively. 
In this case, evolution provenance keeps track of the set of visual analysis steps performed and thereby construct the ``full story'' of a visual analysis session.
For that, we have implemented our evolution provenance model presented in Section~\ref{subsec:evo} to track the visual analytics process of large dynamic graphs. 
%Using the volume-based large dynamic graph analysis solution~\cite{Bruder2019},
%users are engaged in a \emph{exploration session} where they perform various \emph{exploration steps} iteratively. 
%In this case, evolution provenance keeps track of the set of exploration steps performed and thereby construct the ``full story'' of a exploration session.
%For that, we have adjusted slightly our evolution provenance model presented in Section~\ref{subsec:evo} to track the visual analytics process of large dynamic graphs. 



{\color{Fuchsia}In what follows, we detail how the volume-based large dynamic graph analysis solution~\cite{Bruder2019} incorporates the evolution provenance data model presented in Section~\ref{sec:evo-core}. To do that, we define necessary analogous concepts necessary to define the evolution provenance.}


\begin{definition}[Visual analysis step] 
Given an initial dynamic graph $G_i$ that contains a set of timesteps $\{t_1..t_i\}$, we define a visual analysis step as $\exploreStep$ where a dynamic graph $G_s$  that contains $\{t_i..t_j\}$ timesteps  (where $1 \leq i \leq j \leq k $) is visualized using a set of views $V_s=\{v_1..v_p\}$.
\end{definition}


Notice that view is a generic term that covers visualization structures possibly obtained using \emph{data views} analytic method.
In our current implementation, the evolution provenance collector tracks visual analysis steps encompassing the volumetric representation as our main view.
For future work, we also plan to integrate tracking of slice views and the timeline plot.

%\begin{table}[t]
%\taburowcolors[2]{white .. black!10}
%\sffamily\footnotesize
%\tabulinesep=4pt
%\begin{tabu}{|X[cm]|X[cm]|X[cm]|}
%\hline
%\rowcolor{black!80} \color{white}Operation type &   \color{white}Parameters& \color{white}Output\\
%Selection&  $\{t_{i'}..t_{j'}\}$; selected range of timesteps & $G'_{s}=\{t_{i'}..t_{j'}\}$,  $V'_{s}=\{v'_{1}..v'_{p}\}$\\
%Partition&$\{m_1..m_y\}$; the set of split marks & $G'_{s}=[\{t_1..t_m\},\{t_{m+1}..t_l\},..]$ ,  $V'_{s}=\{v'_{1}..v'_{p}\}$ \\
%Aggregation&    $l$, where $l$ is a level & $G_{s}$,  $V'_{s}=\{v'_{1}..v'_{p}\}$ \\
%Filtering&    $cond$, where $cond$ is a predicate& $G_{s}$,  $V'_{s}=\{v'_{1}..v'_{p}\}$ \\
%Color mapping&  $d \times rgb $, where $rgb$ is color and $d$ a graph property &$G_{s}$,  $V'_{s}=\{v'_{1}..v'_{p}\}$ \\
%Camera configuration&  $configuration$   & $G_{s}$, $V'_{s}=\{v'_{1}..v'_{p}\}$ \\
%\hline
%\end{tabu}
%\caption{Permitted analytics operations~\cite{Bruder2019}}
%\label{table:ops}
%\end{table}



\begin{table}[t]
 \centering \scriptsize
 \begin{tabular}{|p{2.5cm}|p{4cm}|p{3cm}|} \hline
\textbf{Operation type}  & \textbf{Parameters} & \textbf{Output}  \\ \hline
Selection&  $\{t_{i'}..t_{j'}\}$; selected range of timesteps & $G'_{s}=\{t_{i'}..t_{j'}\}$,  $V'_{s}=\{v'_{1}..v'_{p}\}$ \\ \hline
Partition&$\{m_1..m_y\}$; the set of split marks & $G'_{s}=[\{t_1..t_m\},\{t_{m+1}..t_l\},..]$ ,  $V'_{s}=\{v'_{1}..v'_{p}\}$  \\ \hline
Aggregation&    $l$, where $l$ is a level & $G_{s}$,  $V'_{s}=\{v'_{1}..v'_{p}\}$ \\ \hline
Filtering&    $cond$, where $cond$ is a predicate& $G_{s}$,  $V'_{s}=\{v'_{1}..v'_{p}\}$  \\ \hline
Color mapping&  $d \times rgb $, where $rgb$ is color and $d$ a graph property &$G_{s}$,  $V'_{s}=\{v'_{1}..v'_{p}\}$  \\ \hline
Camera configuration&  $configuration$   & $G_{s}$, $V'_{s}=\{v'_{1}..v'_{p}\}$  \\ \hline
\end{tabular}
\caption{Permitted analytics operations~\cite{Bruder2019}}
\label{table:ops}
 \end{table}

Table~\ref{table:ops} summarizes supported analytical operations enabling the navigation from a visual analysis step $\exploreStep$ to another step $\exploreStepDest$.
It presents also the set of parameters needed for each operation as well as the structure of the output analysis step $S'$.
Essentially, we distinguish six operation types that handle various views seen over several analysis steps.
All proposed operations to track in evolution provenance are derived from the three fundamental analytic methods thoroughly discussed in Section~\ref{A02:overview} that are data views, filtering/aggregation and comparison.


The \emph{selection function} is associated to the timeline plot feature where the user gets a 2D-visualization whose x-axis depicts a set of timesteps $\{t_i..t_j\}$.
Using this function, the user can select a single or a range of timesteps $\{t_{i'}..t_{j'}\}$  with $t_i \leq t_{i'} \leq t_{j'} \leq t_{j}$ to analyze them visually in the next step.

The \emph{partition operation} corresponds to the volume partitioning analysis feature 
where the user specifies interactively some split marks $\{m_1..m_y\}$ to split the timesteps $\{t_1..t_i\}$ associated to a visual analysis step $S$ into sub-ranges $[\{t_1..t_m\},\{t_{m+1}..t_l\},\ldots]$. 


The evolution provenance model implemented for the volume-based large dynamic graph analysis solution encompasses also the \emph{aggregation operation} and the \emph{filtering operation} (via opacity). 
The former aggregates to a specific level $l$ to alleviate the complexity of a view $V$, while the latter operation omits information available in the current view that do not satisfy a predicate $cond$. As shown in Table~\ref{table:ops}, the two aforementioned operations introduce changes only on the set of views to analyze in the step $S'$ in comparison to step $S$ while keeping the same dynamic graph $G_s$.

The list of analytics operations recorded by our provenance model also contains the \emph{color mapping operation} where a user maps a graph property $d$ (e.g., weights of edges) to a specific range of colors $rgb$ to produce new views $V'_{s}=\{v_{i'}..v_{j'}\}$ for the same dynamic graph $G_s$, seen in the analysis step $S$. 
Finally, we record selected \emph{camera configurations}, where the user selects a certain zoom level, rotation and panning of the camera to get new views $V'_s$ in the next analysis step $S'$.

%Overall, the evolution provenance is modelled by an analysis session graph that gathers all visual analysis steps made by the analyst. 
%Figure~\ref{fig:expo-session} shows an example of such an analysis session graph (augmented with exemplary screenshots of the analytics step), defined as follows.
%
%\begin{figure}[t]
%	\begin{center}
%	\includegraphics[scale=0.35]{figures/evoDM/sessionGraph-crop}
%			\end{center}
%			\caption{Example of an analysis session graph, augmented with images of the respective analytics steps~\cite{Bruder2019}}
%			\label{fig:expo-session}
%\end{figure}
%
%\begin{definition}[Analysis session graph]
%An analysis session graph summarizes user's manipulations over a large Dynamic Graph $D_i$ where $i$ refers to the set of timesteps.
%The analysis session graph is a labeled directed acyclic graph (DAG) $\sessionGraph{}_{D_i}(\sessionV{}, \sessionE{})$ where $\sessionV{}$ is a set of nodes and $\sessionE{}$ a set of labeled edges. 
%Each node $n \in \sessionV$ corresponds to a visual analytics step $S$.
%An edge $e = (n, n', \sessionL{})$ represents the transition from one visual analytics step $S = \{{G_s}, V_s\}$ to the next visual analytics step $S' = \{{G'_s}, V'_s\}$. $\sessionL{}$ is a pair $\langle \labelOp{}, param \rangle$ where $\labelOp{}$ is an identifier of the analytical operation type (see Table~\ref{table:ops}) and $param$ is the set of parameters used to navigate from $S$ to $S'$.
%\label{def:sessionA02}
%\end{definition}

\begin{figure}[t]
	\center
	\includegraphics[scale=0.35]{figures/evoDM/sessionGraph-crop}
			\caption{Example of an analysis session graph, augmented with images of the respective analytics steps~\cite{Bruder2019}}
			\label{fig:expo-session}
\end{figure}
{\color{Fuchsia}Overall, the evolution provenance is implemented in the volume-based large dynamic graph analysis solution~\cite{Bruder2019} following our proposed data model specified in Definition~\ref{def:session}.
Accordingly, the evolution provenance of the volume-based large dynamic graph analysis solution~\cite{Bruder2019} is a graph that gathers all visual analysis steps (analogous to exploration steps) made by the analyst. 
Figure~\ref{fig:expo-session} shows an example of such an analysis session graph (augmented with exemplary screenshots of the analytics step). }%, defined as follows.

Essentially, nodes of the evolution provenance graph correspond to the set of visual analytics steps and edges represents the transition from one visual analytics step to another. %visual analytics step. 
Note that, the labels of edges contain so far only type of navigation and set of parameters used to navigate. In the future, we intend to implement the score $s(e)$ (specified in Definition~\ref{def:session}) that reflects the importance of each edge present in the evolution provenance graph.



\section{Conclusion}
In this chapter, we formalized the evolution provenance model adopted in our visual data exploration framework \framework{}. 
Our novel evolution provenance model ensures the collection of visualization-related properties as well as query-related properties inferred through visual data exploration processes.
It is worth stressing that the evolution provenance is the basis on which recommendations shall be computed in our framework. Accordingly, we show in the next chapter how such provenance can indeed be used for recommendations.
This is done by devising algorithms specialized to the exploration of data stored in data warehouses.


Finally, we point out that our evolution provenance model was also adopted in the context of interactive analysis of dynamic graphs. 
 Further details about the incorporation of our evolution provenance model in the context of dynamic graphs are available in~\cite{Bruder2019}.




