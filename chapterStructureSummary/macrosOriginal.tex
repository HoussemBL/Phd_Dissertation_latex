
\newcommand{\json}{JSON}
\newcommand{\obsol}[1]{\textcolor{gray}{#1}}


%\newtheorem{theorem}{Theorem}[section]
%\newtheorem{conjecture}[theorem]{Conjecture}
%\newtheorem{corollary}[theorem]{Corollary}
%\newtheorem{proposition}[theorem]{Proposition}
%\newtheorem{lemma}[theorem]{Lemma}
%\newdef{definition}[theorem]{Definition}
%\newdef{remark}[theorem]{Remark}
%\newtheorem{example}[theorem]{Example}

\newcommand{\twolines}[2]{\begin{array}{l} #1 \\  #2  \end{array}}


%%\newcommand{\comment}[1]{} %%incompatible with acm style
%\newcommand{\todo}[1]{}
\newif{\ifMarginalComments}
% uncomment one or other of the next two lines
 %\MarginalCommentstrue   %show marginal comments
 \MarginalCommentsfalse  %suppress marginal comments
\newcommand{\Margin}[1]{%
  \ifMarginalComments{\marginpar{\tiny #1}} \else {} \fi}

\newcommand{\DC}[1]{%
  \ifMarginalComments{\marginpar{\tiny[Dario: #1]}} \else {} \fi}

\newcommand{\CS}[1]{%
  \ifMarginalComments{\marginpar{\tiny[Carlo: #1]}} \else {} \fi}
  



  
% ---------------------------------------------------------------------
%
% Proof rule.
%
% ---------------------------------------------------------------------
%

\newcommand{\staterule}[3]{%
  $\begin{array}{@{}l}%
   \mbox{#1}\\%
   \begin{array}{l}
   #2\\
   \hline
   \raisebox{0ex}[2.5ex]{\strut}#3%
   \end{array}
  \end{array}$}

\newlength{\mygap}
\setlength{\mygap}{0.8ex}
\newlength{\smallgap}
\setlength{\smallgap}{0.3ex}
\newcommand{\GAP}{2ex}


% ---------------------------------------------------------------------
%
% Horizontal brackets.
%
% ---------------------------------------------------------------------
%

%Amine: the object and array types
\newcommand{\eset}{\emptyset}

\newcommand{\emprec}{{ \{ \} }}
\newcommand{\emplis}{{ [\ ]}}

\newcommand{\emptrec}{{ \{ \} }}
\newcommand{\emptlis}{{[\ ]}}



\newcommand{\hbra}{
\hbox to \linewidth{\vrule width0.3mm height 1.8mm depth-0.3mm
                    \leaders\hrule height1.8mm depth-1.5mm\hfill
                    \vrule width0.3mm height 1.8mm depth-0.3mm}}
\newcommand{\hket}{
\hbox to \linewidth{\vrule width0.3mm height1.5mm
                    \leaders\hrule height0.3mm\hfill
                    \vrule width0.3mm height1.5mm}}

% --------------------------------------------------------------------- %
% Typesetting definitions:              Sample output:                  %
%                                                                       %
% \begin{defn}                                                          %
% \category{M,N}{terms}\\               M, N ::=        terms           %
% \entry{x}{variable}\\                   x               variable      %
% \entry{M\ N}{application}\\             M N             application   %
% \entry{\lambda x.\ M}{abstraction}      \x.M            abstraction   %
% \end{defn}                                                            %
%                                                                       %
% This is a tabbing environment; the last entry should have no \\.      %
% --------------------------------------------------------------------- %

\newlength{\entryindent}
\newlength{\reductionind}
\newlength{\clausetextind}
\newlength{\entrytextind}
\newlength{\labelind}
\setlength{\entryindent}{.05\linewidth}
\setlength{\reductionind}{.28\linewidth}
\setlength{\clausetextind}{.31\linewidth}
\setlength{\entrytextind}{.36\linewidth}
\setlength{\labelind}{.75\linewidth}

\newlength{\entrytext}
\newlength{\clausetext}

\setlength{\entrytext}{\linewidth}
\addtolength{\entrytext}{-\entrytextind}
\setlength{\clausetext}{\linewidth}
\addtolength{\clausetext}{-\clausetextind}


\newlength{\tabone}
\newlength{\tabtwo}
\newlength{\tabthree}
\newlength{\tabfour}
\newlength{\tabfive}

\setlength{\tabone}{\entryindent}
\setlength{\tabtwo}{\reductionind}
\addtolength{\tabtwo}{-\entryindent}
\setlength{\tabthree}{\clausetextind}
\addtolength{\tabthree}{-\reductionind}
\setlength{\tabfour}{\entrytextind}
\addtolength{\tabfour}{-\clausetextind}
\setlength{\tabfive}{\labelind}
\addtolength{\tabfive}{-\entrytextind}

\newcommand{\ratio}{.18}

%\newlength{\entrytext}
%\newlength{\clausetext}

\newcounter{displayCounter}[section]

\newcommand{\refDisplay}[2]{\ref{#1}.\ref{#2}}
\newcommand{\refDisplayHere}[1]{\thesection.\ref{#1}}

\newcommand{\labelDisplay}[1]{\refstepcounter{displayCounter}
                           \label{#1}
                           \addtocounter{displayCounter}{-1}}

\newenvironment{display}[1]{
  \refstepcounter{displayCounter}
  \setlength{\entrytext}{\linewidth}
  \addtolength{\entrytext}{-\ratio\linewidth}
  \addtolength{\entrytext}{-3em}
  \setlength{\clausetext}{\entrytext}
  \addtolength{\clausetext}{1.5em}
\begin{tabbing}
%\hspace{\tabone}\=\hspace{\tabtwo}\=\hspace{\tabthree}\=\hspace{\tabfour}\=\hspace{\tabfive}\=\kill
  \hspace{1.5em}\=\hspace{\ratio\linewidth}\=\hspace{1.5em}\= \kill
  {Table \thesection.\thedisplayCounter. \emph{#1}}\\[-.8ex]
  \hbra\\[-.8ex]
  }{\\[-.8ex]\hket
  \end{tabbing}}
\newenvironment{displayStar}{\begin{tabbing}
  \hspace{1.5em} \= \hspace{.40\linewidth - 1.5em} \= \hspace{1.5em} \= \kill
  \hbra\\[-.8ex]
  }{\\[-.8ex]\hket
  \end{tabbing}}


%incompatible with acm style
%%\comment{
%%\newenvironment{display}[1]{
%%  \setlength{\entrytext}{\linewidth}
%%  \addtolength{\entrytext}{-\ratio\linewidth}
%%  \addtolength{\entrytext}{-3em}
%%  \setlength{\clausetext}{\entrytext}
%%  \addtolength{\clausetext}{1.5em}
%%\begin{tabbing}
%%%\hspace{\tabone}\=\hspace{\tabtwo}\=\hspace{\tabthree}\=\hspace{\tabfour}\=\hspace{\tabfive}\=\kill
%%  \hspace{1.5em}\=\hspace{\ratio\linewidth}\=\hspace{1.5em}\= \kill
%%  \textbf{#1}\\[-.8ex]
%%  \hbra\\[-.8ex]
%%  }{\\[-.8ex]\hket
%%  \end{tabbing}}
%%\newenvironment{displayStar}{\begin{tabbing}
%%  \hspace{1.5em} \= \hspace{.40\linewidth - 1.5em} \= \hspace{1.5em} \= \kill
%%  \hbra\\[-.8ex]
%%  }{\\[-.8ex]\hket
%%  \end{tabbing}}
%%}%end comment

\newcommand{\entry}[2]{\>$#1$\>\>\parbox[t]{\entrytext}{#2\vspace{1mm}}}
\newcommand{\fentry}[3]{\>$#1$\>\>#2\>\parbox[t]{\entrytext}{#3\vspace{1mm}}}
\newcommand{\clause}[2]{$#1$\>\>\parbox[t]{\clausetext}{#2\vspace{1mm}}}
%\newcommand{\eqrule}[2]{$#1$\>#2}
\newcommand{\eqrule}[2]{$#1$\> }
\newcommand{\redrule}[3]{$#1$\>\>$#2$\>\>\>#3}
\newcommand{\redclause}[3]{$#1$\>$#2$\>$\to$\ \ $#3$}
\newcommand{\defclause}[3]{$#1$\>$#2$\>$\eqdef$\ \ $#3$}
\newcommand{\eqclause}[3]{$#1$\>$\eqdef$\ \ ${#2}$\>$#3$}
\newcommand{\noeqclause}[3]{$#1$\>\ \ \qquad \ \ $#2$\>$#3$}
\newcommand{\iffclause}[3]{$#1$\>$#2$\>$\Iff$\ \ $#3$}
\newcommand{\noiffclause}[2]{$#1$\>$#2$}
\newcommand{\xcategory}[2]{\clause{#1::=}{#2}}
\newcommand{\fcategory}[2]{$#1::=$\>\>#2}
\newcommand{\subclause}[1]{\>\>\>\>#1}

\newcommand{\iflongmath}[1]{}
\newcommand{\ifshortmath}[1]{#1}
\newcommand{\ifveryshort}[1]{}
\newcommand{\iftimes}[1]{}
\newcommand{\ifshort}[1]{#1}
\newcommand{\iflong}[1]{}
\newcommand{\ifboth}[1]{#1}


\newcommand{\mr}{Map/Reduce}
\newcommand{\map}{Map}
\newcommand{\redc}{Reduce}
\newcommand{\pair}[2]{< {#1} ; {#2} >}
\newcommand{\sub}{<:}
%\newcommand{\rec}[2]{{\tt\bf\{}#1 , \ldots , #2\}}
\newcommand{\rec}[2]{\{#1 , \ldots , #2\}}
\newcommand{\arr}[2]{[#1 , \ldots , #2]}
\newcommand{\str}{s}
\newcommand{\chr}{c}
\newcommand{\num}{n}
\newcommand{\nule}{null}
\newcommand{\rectype}[2]{\{#1 , \ldots , #2\}}
\newcommand{\arrtype}[2]{[#1 , \ldots , #2]}
\newcommand{\srectype}[1]{\{ #1 \}}
\newcommand{\seqT}{T, \ldots , T}
\newcommand{\arrytype}[1]{\cdot (#1)}
\newcommand{\uniontype}[1]{+(#1)}
\newcommand{\rconc}{\circ}
\newcommand{\tconc}{\cdot}
\newcommand{\tunion}{\boldsymbol{\scriptscriptstyle +}}
\newcommand{\ssem}[1]{\llbracket #1 \rrbracket}
\newcommand{\eqdef}{\ensuremath{\stackrel{\vartriangle}{=}}}
\newcommand{\noteq}{\not=}
\newcommand{\base}{\textsf{B}}
\newcommand{\Set}[1]{\{#1\}}
\newcommand{\strequiv}{\simeq}
\newcommand{\subt}{\lesssim}
\newcommand{\frule}[3]{#1 \mid #2 \rightarrow #3}
\newcommand{\field}[1]{#1}
\newcommand{\lbl}[1]{lbl(#1)}
\newcommand{\sproj}{\subt}
\newcommand{\Union}{\tunion}
\newcommand{\Aunion}{\dotplus}%%Amine
\newcommand{\mc }[1]{{\mathcal #1 }}
\newcommand{\rectln}{\rectype{l_{1}:T_{1}}{l_{n}:T_{n}}}
\newcommand{\rectlnp}{\rectype{l_{1}:T_{1}, \ldots , l_{i}:T_{i}^{\prime}, l_{i+1}:T_{i+1}}{l_{n}:T_{n}}}
\newcommand{\recs}[2]{\srectype{#1:#2}}
\newcommand{\recss}[2]{(#1, #2)} %%Amine: alternative
\newcommand{\rop}{\bullet}
\newcommand{\quant}[2]{#2^{#1}}
\newcommand{\iquant}[1]{\quant{1}{#1}}
\newcommand{\oquant}[1]{\quant{?}{#1}}
\newcommand{\pquant}[1]{\quant{+}{#1}}
\newcommand{\squant}[1]{\quant{*}{#1}}
\newcommand{\quantc}{\oast}
\newcommand{\nobj}{N_{obj}}
\newcommand{\tin}{T_{in}}
\newcommand{\perr}{P_{err}}
\newcommand{\dist}{dist}

\newcommand{\false}{\textit{\bf False}}
\newcommand{\true}{\textit{\bf True}}
\newcommand{\isarr}[1]{\text{IsArr}(#1)}
\newcommand{\isrec}[1]{\text{IsRec}(#1)}
\newcommand{\labels}[1]{\text{Labels}(#1)}

\newcommand{\fieldd}[2]{#1{:}#2}
\newcommand{\fieldt}[2]{(#1{:}#2)}

%%%%%% JUDGMENTS

\newcommand{\judbase}[2]{\vdash \; #1 \; : \; #2}
\newcommand{\judbasee}[2]{\vdash  #1  \leadsto  #2}

\newcommand{\judwf}[1]{\vdash_{WF} \; #1 }


%% ALGORITHMS

\newcommand{\infc}{\textsc{INF}$_{C}$}
\newcommand{\infcmr}{\textsc{INF}$_{CMR}$}
\newcommand{\infmr}{\textsc{INF}$_{MR}$}

\newcommand{\rmark}[2]{[\textbf{#1}:#2]}
\newcommand{\rmarkCS}[1]{\rmark{CS}{#1}}
\newcommand{\rmarkDC}[1]{\rmark{DC}{#1}}


%%%%Boolean functions %%%%%


\newcommand{\N}[1]{\text{N}(#1)}

\newcommand{\biga}{{\cal A}}
\newcommand{\bigb}{{\cal B}}

\newcommand{\bigt}{{\cal T}}
\newcommand{\bigu}{{\cal U}}
\newcommand{\bigv}{{\cal V}}
\newcommand{\bigw}{{\cal W}}
\newcommand{\bigp}{{\cal P}}
\newcommand{\bigrr}{{\cal R}}

\newcommand{\bigunion}{}

\newcommand{\ft}{{\tt \bf t}}
\newcommand{\fu}{{\tt \bf u}}
\newcommand{\fv}{{\tt \bf v}}
\newcommand{\fw}{{\tt \bf w}}

\newcommand{\interp}[1]{{\tt Perm}(#1)}




%%%%%%%%AMINE NEW %%%%%%%
\newcommand{\listfusion}[1]{\llistfusion(#1)}
\newcommand{\llistfusion}{LFuse}

\newcommand{\fusion}[2]{\lfusion(#1 \mid #2)}
\newcommand{\lfusion}{Fuse}

\newcommand{\afusion}[2]{\lafusion(#1 \mid #2)}
\newcommand{\afusionp}[3]{\lafusion^{#3}(#1 \mid #2)}
\newcommand{\lafusion}{U\!Fuse}

\newcommand{\opt}{{\bf opt}}


\newcommand{\ofusion}[2]{\lofusion(#1 \mid #2)}
\newcommand{\lofusion}{F\!Fuse}


\newcommand{\isbas}[1]{{isBasic}(#1)}

\newcommand{\collaps}[1]{\lcollaps(#1)}
\newcommand{\lcollaps}{ASimp}

\newcommand{\collapss}[2]{\lcollapss(#1\mid#2)}
\newcommand{\lcollapss}{ASimpAcc}


\newcommand{\forder}{\prec}
\newcommand{\tnull}{{\tt Null}}
\newcommand{\tbool}{{\tt Bool}}
\newcommand{\tnum}{{\tt Num}}
\newcommand{\tstr}{{\tt Str}}

\newcommand{\flatn}[1]{Flatten(#1)}
%\newcommand{\norml}[1]{Normalize(#1)}
\newcommand{\mmul}{{\tt m}}
\newcommand{\nmul}{{\tt n}}
\newcommand{\mini}[2]{min(#1,#2)}

\newcommand{\card}[2][]{{\mid}{#2}{\mid_{\scriptsize #1}}}
\newcommand{\Head}[1]{head(#1)}
\newcommand{\Tail}[1]{tail(#1)}


\newcommand{\kind}[1]{kind(#1)}
\newcommand{\xmatch}[2]{\lxmatch(#1 \mid #2)}
\newcommand{\lxmatch}{KMatch}

\newcommand{\xunmatch}[2]{\lxunmatch(#1\mid#2)}
\newcommand{\lxunmatch}{KUnmatch}

\newcommand{\rmatch}[2]{\lrmatch(#1\mid#2)}
\newcommand{\lrmatch}{FMatch}

\newcommand{\runmatch}[2]{\lrunmatch(#1\mid#2)}
\newcommand{\lrunmatch}{FUnMatch}


\newcommand{\rextract}[1]{\lrextract(#1)}
\newcommand{\lrextract}{RExtract}


\newcommand{\bigplus}[2]{\stackrel[#1]{#2}{\oplus}}
%\newcommand{\bigun}[2]{\stackrel[#1]{#2}{\bigcup}}
\newcommand{\bigun}[2]{\stackrel[#1]{#2}{\bigcirc}}


%%color background to highlight new version
\newcommand{\nversion}[1]{\begin{mdframed}[backgroundcolor=blue!20]
#1\end{mdframed}
}


\newcommand{\regle}[3][]{\frac{\begin{array}{c}\displaystyle#2\end{array}}{\displaystyle#3}{\ \ {\scriptstyle{#1}}}} 	

